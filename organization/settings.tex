%
%
% Setups
%
\interfootnotelinepenalty=6000 %Gibt die Dringlichkeit an, mit der Fussnoten
					%nicht umgebrochen werden (bis 10000)
					%Standard: 100
%
% \newcolumntype{L}{>{\raggedright\arraybackslash}X}
%\setlength{\parindent}{0em}
%\setlength{\parskip}{0ex plus 0pt minus 0pt}
% Default value is 0pt plus 1pt
%Latex Std is \setlength\parskip{0\p@ \@plus \p@}
%
%
%\tikzset{
%	fontscale/.style = {font=\relsize{#1}}
%	}
%\counterwithin{figure}{section}
% PSTricks standart umgebung:
%\psset{xunit=0.5\textwidth,yunit=0.5\textwidth,runit=0.5\textwidth}
%
%
%
%
%
%________________________________________________________________________
%------------------------------------------------------------------------
%		Spacing für
%					- Equation Environment
%					- floats
%					- Some new Lengths & Variables
%/\/\/\/\/\/\/\/\/\/\/\/\/\/\/\/\/\/\/\/\/\/\/\/\/\/\/\/\/\/\/\/\/\/\/\/\
%	Is found in
%		%
%================================================================
%----------------------------------------------------------------
%      Setup, Typographisch
%             (Für Präambel. Titel, Verzeichnisse...)
%================================================================
%
\pagestyle{empty}%
% Hm, settings vlt. doch besser in Präambel
% %
%
% Setups
%
\interfootnotelinepenalty=6000 %Gibt die Dringlichkeit an, mit der Fussnoten
					%nicht umgebrochen werden (bis 10000)
					%Standard: 100
%
% \newcolumntype{L}{>{\raggedright\arraybackslash}X}
%\setlength{\parindent}{0em}
%\setlength{\parskip}{0ex plus 0pt minus 0pt}
% Default value is 0pt plus 1pt
%Latex Std is \setlength\parskip{0\p@ \@plus \p@}
%
%
%\tikzset{
%	fontscale/.style = {font=\relsize{#1}}
%	}
%\counterwithin{figure}{section}
% PSTricks standart umgebung:
%\psset{xunit=0.5\textwidth,yunit=0.5\textwidth,runit=0.5\textwidth}
%
%
%
%
%
%________________________________________________________________________
%------------------------------------------------------------------------
%		Spacing für
%					- Equation Environment
%					- floats
%					- Some new Lengths & Variables
%/\/\/\/\/\/\/\/\/\/\/\/\/\/\/\/\/\/\/\/\/\/\/\/\/\/\/\/\/\/\/\/\/\/\/\/\
%	Is found in
%		%
%================================================================
%----------------------------------------------------------------
%      Setup, Typographisch
%             (Für Präambel. Titel, Verzeichnisse...)
%================================================================
%
\pagestyle{empty}%
% Hm, settings vlt. doch besser in Präambel
% \input{./organization/settings.tex}%
\microtypesetup{activate=false}%
%%%%%%%%%%%%%%%%%%%%%%%%%%%%%%%%%%%%%%%%%%%%%%%%%%%%%%%%%%%%%%%%%
% Tracking, Spacing & Kerning werden seit Lua(La)Tex von 'fontspec' verwaltet
% 				% \microtypesetup{tracking=false}%
% 				% \microtypesetup{kerning=false}%
% 				% \microtypesetup{spacing=false}%
%
%================================================================
%----------------------------------------------------------------
%      Title Page \& Schmutztitel
%================================================================
%
\input{./organization/TitlePageData.tex}%
%\maketitle
\input{./chapter/supplement/title_plain.tex}%
%
% Title Page Rückseite
%
\vspace*{\fill}%
\begin{flushright}%
\begin{large}%
\TPauthorFirst\ \TPauthorLast\nl%
% v0.1 - 02.09.2014\nl
% v1.0 - 03.11.2014\nl
% v2.0 - 28.01.2015\nl
% Final - 28.01.2015\nl
% aktuell: \today
\TPdate%
\end{large}%
\end{flushright}%
\clearpage%
%
%================================================================
%----------------------------------------------------------------
%      Inhaltsverzeichnis
%================================================================
%
\begingroup%
  \renewcommand*{\chapterpagestyle}{empty}%
  \pagestyle{plain}%
  \tableofcontents%
%   \thispagestyle{empty}%
  \clearpage%
%   \listoffigures%
%   \clearpage%
%   \listoftables%
%   \clearpage%
\endgroup%
%
%================================================================
%----------------------------------------------------------------
%      Setup, Typographisch
%             (Für eigentliches Dokument)
%================================================================
%
% \selectlanguage{ngerman}%
%
%\sloppy % Würde die Dehnung der Zwischenräume etwas lockerer machen
%\fussy % Schaltet \sloppy wieder aus. Ist der Latex-Standart
\microtypesetup{activate=true}%
%%%%%%%%%%%%%%%%%%%%%%%%%%%%%%%%%%%%%%%%%%%%%%%%%%%%%%%%%%%%%%%%%
% Tracking, Spacing & Kerning werden seit Lua(La)Tex von 'fontspec' verwaltet
% 					% \microtypesetup{tracking=true}%
% 					% \microtypesetup{kerning=true}%
% 					% \microtypesetup{spacing=true}%
%
% Beachte: Typographisch erhalten Kapitelseiten keinen Kolumnentitel: plain
\renewcommand*{\chapterpagestyle}{plain.scrheadings}%
\pagestyle{scrheadings}%
\renewcommand*{\partpagestyle}{empty}%
%
%
%
%
%
%________________________________________________________________________
%------------------------------------------------------------------------
%		Spacing für
%					- Equation Environment
%					- floats
%					- Some new Lengths & Variables
%/\/\/\/\/\/\/\/\/\/\/\/\/\/\/\/\/\/\/\/\/\/\/\/\/\/\/\/\/\/\/\/\/\/\/\/\
% - - - - - - - - - - - - - - - - - - - - - - - - - - - - - - - - - - - -
%- - - - - - - - - - - - - - - - - - - - - - - - - - - - - - - - - - - - -
% - - - - - - - - - - - - - - - - - - - - - - - - - - - - - - - - - - - -
%		Equation
% - - - - - - - - - - - - - - - - - - - - - - - - - - - - - - - - - - - -
%------------------------------------------------------------------------
% 	\begin{equation}\begin{split}
% 	MRes_{K}^{3rd} = \delta_{3rd} + \delta_{K}
% 	\end{split}\end{equation}
%------------------------------------------------------------------------
% Default for 'above' & 'below' Space:
%		11.0pt plus 3.0pt minus 6.0pt
% You can check this using (prints it into the Document):
%		\the\abovedisplayskip
%		\the\belowdisplayskip
%------------------------------------------------------------------------
% The ones with 'short' come in play if the last line immediately before an equation is, well, short.
%/\/\/\/\/\/\/\/\/\/\/\/\/\/\/\/\/\/\/\/\/\/\/\/\/\/\/\/\/\/\/\/\/\/\/\/\
\setlength{\abovedisplayskip}{3pt}
\setlength{\belowdisplayskip}{3pt}
\setlength{\abovedisplayshortskip}{0pt}
\setlength{\belowdisplayshortskip}{0pt}
% - - - - - - - - - - - - - - - - - - - - - - - - - - - - - - - - - - - -
%- - - - - - - - - - - - - - - - - - - - - - - - - - - - - - - - - - - - -
% - - - - - - - - - - - - - - - - - - - - - - - - - - - - - - - - - - - -
%		Floats
% - - - - - - - - - - - - - - - - - - - - - - - - - - - - - - - - - - - -
\setlength\extrarowheight{1ex}%
% The Spacing between floats and the surrounding text at top and bottom of the
% float. The Defaults are somewhat complicated and depend on document type and
% font-size%
% \setlength{\intextsep}{\intextsep}%
% Some more lenghts like this
%     \textfloatsep — distance between floats on the top or the bottom and the text;
%     \floatsep — distance between two floats;
%     \intextsep — distance between floats inserted inside the page text (using h) and the text proper.
% - - - - - - - - - - - - - - - - - - - - - - - - - - - - - - - - - - - -
%- - - - - - - - - - - - - - - - - - - - - - - - - - - - - - - - - - - - -
% - - - - - - - - - - - - - - - - - - - - - - - - - - - - - - - - - - - -
%		New Lenghts & Variables
% - - - - - - - - - - - - - - - - - - - - - - - - - - - - - - - - - - - -
\newlength\textheighttemp%
% \setlength{\textheighttemp}{\textheight}%
\newlength\textwidthtemp%
% \setlength{\textwidthtemp}{\textwidth}%
\newlength\textheightstd%
\setlength{\textheightstd}{\textheight}%
\newlength\textwidthstd%
\setlength{\textwidthstd}{\textwidth}%
\newlength\textheightold%
\newlength\textwidthold%
%
\newlength\tempheight%
\newlength\tempwidth%
%\/\/\/\/\/\/\/\/\/\/\/\/\/\/\/\/\/\/\/\/\/\/\/\/\/\/\/\/\/\/\/\/\/\/\/\/
%			Spacing End
%------------------------------------------------------------------------
%________________________________________________________________________
%
% ( '\setlength' must come after \begin{document} )
%/\/\/\/\/\/\/\/\/\/\/\/\/\/\/\/\/\/\/\/\/\/\/\/\/\/\/\/\/\/\/\/\/\/\/\/\
%			Spacing End
%------------------------------------------------------------------------
%________________________________________________________________________
%
%
%
%
%
%
%
%
%
%________________________________________________________________________
%------------------------------------------------------------------------
%							Setups für Microtype
%/\/\/\/\/\/\/\/\/\/\/\/\/\/\/\/\/\/\/\/\/\/\/\/\/\/\/\/\/\/\/\/\/\/\/\/\
% \SetProtrusion{encoding={*},family={bch},series={*},size={6,7}}
%               {1={ ,750},2={ ,500},3={ ,500},4={ ,500},5={ ,500},
%                6={ ,500},7={ ,600},8={ ,500},9={ ,500},0={ ,500}}
%%%%%%%%%%%%%%%%%%%%%%%%%%%%%%%%%%%%%%%%%%%%%%%%%%%%%%%%%%%%%%%%%
% Tracking, Spacing & Kerning werden seit Lua(La)Tex von 'fontspec' verwaltet
% 				% \SetExtraKerning[unit=space]
% 				%     {encoding={*}, family={bch}, series={*}, size={footnotesize,small,normalsize}}
% 				%     {\textendash={400,400}, % en-dash, add more space around it
% 				%      "28={ ,150}, % left bracket, add space from right
% 				%      "29={150, }, % right bracket, add space from left
% 				%      \textquotedblleft={ ,150}, % left quotation mark, space from right
% 				%      \textquotedblright={150, }} % right quotation mark, space from left
% 				% \SetTracking{encoding={*}, shape=sc}{40}
%\/\/\/\/\/\/\/\/\/\/\/\/\/\/\/\/\/\/\/\/\/\/\/\/\/\/\/\/\/\/\/\/\/\/\/\/
%							Microtype fertig
%------------------------------------------------------------------------
%________________________________________________________________________
%
%
%
%
%
%
%
%________________________________________________________________________
%------------------------------------------------------------------------
%							Setups für Glossaries
%/\/\/\/\/\/\/\/\/\/\/\/\/\/\/\/\/\/\/\/\/\/\/\/\/\/\/\/\/\/\/\/\/\/\/\/\
%Formatierung für Abkürzungen bei erstem Auftreten und weiteren und
%Falls deutsch Key im GLossar vorhanden, benutze ihn beim ersten Mal
% \defglsentryfmt[\acronymtype]{
%   \ifglsused{\glslabel}
%     {\glsgenentryfmt}% Wenn verwendet normales format
%     {\textit{\glsgenentryfmt}}% beim ersten mal kursiv
% }
% 		\newcommand*{\glsprintmitdeutsch}{%
% 			{\glsentrylong{\glslabel} (\glsentryname{\glslabel}, dt.:
% 			\glsentrydeutsch{gls-\glslabel})}%
% 		}%
% 		\newcommand*{\glsprintmitenglisch}{%
% 			{\glsentrylong{\glslabel} (\glsentryname{\glslabel}, engl.:
% 			\glsentryenglisch{gls-\glslabel})}%
% 		}%
% 		\newcommand*{\glsprintmitdualentry}{%
% 		  \ifglsused{\glslabel}%
% 		    {%
% 		    	\glsgenentryfmt%
% 		    }{%
% 		    	\ifglshasfield{deutsch}{gls-\glslabel}{%
% 		    		\glsprintmitdeutsch%
% 		    	}{%
% 			    	\ifglshasfield{englisch}{gls-\glslabel}{%
% 			    		\glsprintmitenglisch%
% 			    	}{%
% 			    		{\glsgenentryfmt}%
% 			    	}%
% 		    	}%
% 		    }%
% 		}%
% 		\newcommand*{\glsprintohnedeutsch}{%
% 		  \ifglsused{\glslabel}%
% 		    {\glsgenentryfmt}% Wenn verwendet normales format
% 		    {\glsgenentryfmt}% beim ersten mal anders formatiert
% 		}%
% 		% Alternativen für erste Formatierung:
% 		% textsf, textit, texttt
% 		%
% 		\defglsentryfmt[\acronymtype]{%
% 			\ifglsentryexists{gls-\glslabel}{%
% 				\glsprintmitdualentry%
% 			}{%
% 				\glsprintohnedeutsch%
% 			}%
% 		}%
% 		%
% 		\renewcommand*{\glsclearpage}{}%
% 		%
% 		\newlist{myglossarylist}{description}{10}%
% 		\setlist[myglossarylist]{%
% 			align=left,%
% 		% 	itemindent=0em,%
% 		% 	labelindent=0em,%
% 		% 	listparindent=0em,%
% 		}%
% 		\setlist[myglossarylist,2]{%
% 			align=left,%
% 		 	leftmargin=2em,%
% 			itemindent=-2em,%
% 		}%
% 		\newglossarystyle{list+url}
% 		{% based on list style (adapt as required)
% 		  \setglossarystyle{list}%
% 		    \renewcommand{\glossentry}[2]{%
% 		    \item[\glsentryitem{##1}%
% 		          \glstarget{##1}{\glossentryname{##1}}]
% 		       \glossentrydesc{##1}\glspostdescription\space##2%
% 		    \ifglshasfield{url}{##1}{\glspar
% 		     \glsletentryfield{\thisurl}{##1}{url}%
% 		     \expandafter\url\expandafter{\thisurl}}{}}%
% 		}%
% 		\newglossarystyle{list+Deutsch+Untertitel+alternativ+url}{%
% 			\renewenvironment{theglossary}%
% 				{\begin{myglossarylist}}{\end{myglossarylist}}%
% 			\renewcommand*{\glossaryheader}{}%
% 			\renewcommand*{\glsgroupheading}[1]{}%
% 			\renewcommand*{\glossentry}[2]{%
% 				\item[\glsentryitem{##1}%
% 					  \glstarget{##1}{\glossentryname{##1}}]%
% 		%
% 				\ifglshasfield{untertitel}{##1}{\normalfont%
% 					\textbf{(}\glsentryuntertitel{##1}\textbf{)}.\space}{}%
% 				\ifglshasfield{alternativ}{##1}{\normalfont%
% 					Auch genannt: \textbf{\glsentryalternativ{##1}}.\space}{}%
% 				\ifglshasfield{deutsch}{##1}{Zu Deutsch: %
% 					\textbf{\glsentrydeutsch{##1}}.\space}{}%
% 				\glossentrydesc{##1}\glspostdescription\space ##2%
% 		    \ifglshasfield{url}{##1}{\glspar
% 				\url{\glsentryurl{##1}}}{}%
% 			}%
% 			\renewcommand*{\subglossentry}[3]{%
% 				\begin{myglossarylist}%
% 					\item[\glssubentryitem{##2}%
% 					\glstarget{##2}{\strut\glossentryname{##2}%
% 					}]%
% 		%
% 					\ifglshasfield{untertitel}{##2}{\normalfont%
% 						\textbf{(}\glsentryuntertitel{##2}\textbf{)}.\space}{}%
% 					\ifglshasfield{alternativ}{##2}{\normalfont%
% 						Auch genannt: \textbf{\glsentryalternativ{##2}}.\space}{}%
% 					\ifglshasfield{deutsch}{##2}{Zu Deutsch: %
% 						\textbf{\glsentrydeutsch{##2}}.\space}{}%
% 					\glossentrydesc{##2}\glspostdescription\space ##3.%
% 				\end{myglossarylist}%
% 			}%
% 			\renewcommand*{\glsgroupskip}{\ifglsnogroupskip\else\indexspace\fi}%
% 		}%
% 		\setglossarystyle{list}%
%
% \setacronymstyle{long-short}
%/\/\/\/\/\/\/\/\/\/\/\/\/\/\/\/\/\/\/\/\/\/\/\/\/\/\/\/\/\/\/\/\/\/\/\/\
%							Glossaries fertig
%------------------------------------------------------------------------
%________________________________________________________________________
%
%
%
%
%________________________________________________________________________
%------------------------------------------------------------------------
%							Setups für enumitem
%					Listen - (itemize, enumerate, description)
%/\/\/\/\/\/\/\/\/\/\/\/\/\/\/\/\/\/\/\/\/\/\/\/\/\/\/\/\/\/\/\/\/\/\/\/\
% \setlist[1]{\labelindent=\parindent} % < Usually a good idea
\setlist[description]{font=\sffamily\bfseries} % the Default
%/\/\/\/\/\/\/\/\/\/\/\/\/\/\/\/\/\/\/\/\/\/\/\/\/\/\/\/\/\/\/\/\/\/\/\/\
%							enumitem fertig
%------------------------------------------------------------------------
%________________________________________________________________________
%
%
%
%
%
%
%________________________________________________________________________
%------------------------------------------------------------------------
%						Setups für Hyperref/URL
%		Note: Package hyperref internally laods package url
%		(Like some other Packages do. For Example: biblatex)
%/\/\/\/\/\/\/\/\/\/\/\/\/\/\/\/\/\/\/\/\/\/\/\/\/\/\/\/\/\/\/\/\/\/\/\/\
\mathchardef\UrlBreakPenalty=100% I think default is 100
% Allow linebreaks in URLs after additional character to the defaults, but
% don't just \renew or \def it, because this would remove all characters
% predefined within \UrlBreaks in the package. This could mislead other users.
% E.g. can cause smaller dots.. So use the etoolbox with "append to"
\appto\UrlBreaks{%
	\do\*%
% 	\do\-% Is done with the url option "hyphens"
	\do\~%
	\do\"%
	\do\a\do\b\do\c\do\d\do\e\do\f\do\g\do\h\do\i\do\j%
	\do\k\do\l\do\m\do\n\do\o\do\p\do\q\do\r\do\s\do\t\do\u\do\v\do\w%
	\do\x\do\y\do\z%
}%
%	Note: Alread inside the \UrlBreaks by default are:
% 	\do\.%
% 	\do\=%
% 	\do\'%
% 	\do\&%
% 	\do\-% With the option hypens
% \expandafter\def\expandafter\UrlBreaks\expandafter{\UrlBreaks% save the current one
% 	\do\*%
% 	\do\-%
% 	\do\~%
% 	\do\'%
% 	\do\"%
% 	\do\-%
% 	\do\&%
% 	\do\=%
% }%
%/\/\/\/\/\/\/\/\/\/\/\/\/\/\/\/\/\/\/\/\/\/\/\/\/\/\/\/\/\/\/\/\/\/\/\/\
%						Hyperref/URL fertig
%------------------------------------------------------------------------
%________________________________________________________________________
%
%
%
%
%
%
%
%________________________________________________________________________
%------------------------------------------------------------------------
%						Redefinition of \texttt
%				to allow a proper hyphenation (Silbentrennung)
% other breakpoint symbols can be added, for example, below I also make [ a breakpoint:
%/\/\/\/\/\/\/\/\/\/\/\/\/\/\/\/\/\/\/\/\/\/\/\/\/\/\/\/\/\/\/\/\/\/\/\/\
\let\stdtexttt\texttt
% \newcommand{\newtexttt}[1]{%
%   \begingroup
%   \ttfamily
%   \begingroup\lccode`~=`/\lowercase{\endgroup\def~}{/\discretionary{}{}{}}%
%   \begingroup\lccode`~=`[\lowercase{\endgroup\def~}{[\discretionary{}{}{}}%
%   \begingroup\lccode`~=`.\lowercase{\endgroup\def~}{.\discretionary{}{}{}}%
%   \catcode`/=\active\catcode`[=\active\catcode`.=\active
%   \scantokens{#1\noexpand}%
%   \endgroup
% }
% Oh noes, better way: Look at the Schriftart Segment in the Header
%/\/\/\/\/\/\/\/\/\/\/\/\/\/\/\/\/\/\/\/\/\/\/\/\/\/\/\/\/\/\/\/\/\/\/\/\
%						Redef fertig
%------------------------------------------------------------------------
%________________________________________________________________________
%
%
%
%
%
%________________________________________________________________________
%------------------------------------------------------------------------
%							Setups für Cleveref
%/\/\/\/\/\/\/\/\/\/\/\/\/\/\/\/\/\/\/\/\/\/\/\/\/\/\/\/\/\/\/\/\/\/\/\/\
\crefname{chapter}{Kapitel}{Kapiteln}
\crefname{section}{Abschnitt}{Abschnitten}
\crefname{part}{Teil}{Teilen}
\crefname{figure}{Abbildung}{Abbildungen}
\crefname{table}{Tabelle}{Tabellen}
\crefname{listing}{Listing}{Listings}
\crefname{equation}{Formel}{Formeln}
%/\/\/\/\/\/\/\/\/\/\/\/\/\/\/\/\/\/\/\/\/\/\/\/\/\/\/\/\/\/\/\/\/\/\/\/\
%							Cleveref fertig
%------------------------------------------------------------------------
%________________________________________________________________________
%
%
%
%
%
%________________________________________________________________________
%------------------------------------------------------------------------
%						Definitionen allgemeiner Farben
%/\/\/\/\/\/\/\/\/\/\/\/\/\/\/\/\/\/\/\/\/\/\/\/\/\/\/\/\/\/\/\/\/\/\/\/\
\definecolor{optionalargscolor}{RGB}{255,125,25}%
\definecolor{ColorRegExCmdID}{RGB}{25,0,255}%
\definecolor{ColorRegExCmdSelect}{RGB}{255,125,25}%
\definecolor{ColorRegExCmdMultiSelect}{RGB}{30,150,0}%
\definecolor{ColorRegExCmdOption}{RGB}{100,100,100}%
%/\/\/\/\/\/\/\/\/\/\/\/\/\/\/\/\/\/\/\/\/\/\/\/\/\/\/\/\/\/\/\/\/\/\/\/\
%							Farben fertig
%------------------------------------------------------------------------
%________________________________________________________________________
%
%
%
%
%
%
%
%
% Initialisierungen wie Counter, Theorem, Variablen, Konstanten..
%
\theoremstyle{break}
\theoremheaderfont{\bfseries}
\theorembodyfont{\normalfont}
\theoremseparator{}
\theorempreskip{0ex}
\theorempostskip{0ex}
\theoremindent0.5em
%
%
% 
% 
%
%
%
%
%________________________________________________________________________
%------------------------------------------------------------------------
%							listings - Setup. Source-Code Einbindung.
%/\/\/\/\/\/\/\/\/\/\/\/\/\/\/\/\/\/\/\/\/\/\/\/\/\/\/\/\/\/\/\/\/\/\/\/\
\newcommand{\lstC}[1]{\lstinline[language=C,breaklines=true,basicstyle=\listingsfontinline]$#1$}
% Colors etc. to get the Eclipse Look
% \newfontfamily\listingsfont[Scale=0.7]{Courier} 
% \newfontfamily\listingsfontinline[Scale=0.8]{Courier New} 
\definecolor{eclipse_comment}{rgb}{0.12, 0.38, 0.18 } %adjusted, in Eclipse: {0.25, 0.42, 0.30 } = #3F6A4D
\definecolor{eclipse_keyword1}{RGB}{128, 0, 40}  % red
\definecolor{eclipse_keyword2}{RGB}{155,40,128} %purple
\definecolor{eclipse_keyword3}{RGB}{0,120,10} %green
\definecolor{eclipse_string}{rgb}{0.06, 0.10, 0.98} % #101AF9
\def\lstsmallmath{\leavevmode\ifmmode \scriptstyle \else  \fi}
\def\lstsmallmathend{\leavevmode\ifmmode  \else  \fi}
%-----------
\definecolor{lstbg}{RGB}{230,230,230}
\lstset{language=C,texcl=true}
% Basic lstset (optical settings)
% \lstset{
% 	numbers=left,
% 	numberstyle=\tiny,
% 	numbersep=5pt,
% 	tabsize=4,
% 	xleftmargin=0em,
% 	backgroundcolor=\color{lstbg},
% 	keywordstyle=\bfseries\ttfamily\color{blue},
% 	stringstyle=\color{green}\ttfamily,
% 	commentstyle=\color{gray}\ttfamily,
% }
% lstset for Look like Eclipse
\lstset{
	frame=shadowbox,
	numbers=left,
	numberstyle=\tiny,
	numbersep=5pt,
	showspaces=false,
	showtabs=false,
	tabsize=2,
	breaklines=true,
	keepspaces=true,
% 	basicstyle=\listingsfont,
	captionpos=b,
	xleftmargin=0em,
	xrightmargin=0.3em,
	backgroundcolor=\color{lstbg},
	keywordstyle=\color{eclipse_keyword1}\bfseries,
	keywordstyle=[2]\color{eclipse_keyword2}\bfseries,
	keywordstyle=[3]\color{eclipse_keyword3}\bfseries,
	keywordstyle=[9]\color{red}\bfseries,
	stringstyle=\color{eclipse_string},
	commentstyle=\color{eclipse_comment}\itshape,
	escapebegin={\lstsmallmath}, escapeend={\lstsmallmathend}
}
\lstset{%
	inputencoding=utf8,%
	extendedchars=true,%
	literate=%
		{ß}{{\ss}}2%
		{ö}{{\"o}}1%
		{ä}{{\"a}}1%
		{ü}{{\"u}}1%
		{Ö}{{\"O}}1%
		{Ä}{{\"A}}1%
		{Ü}{{\"U}}1%
		{>>}{{\guillemotright}}1%
		{»}{{\guillemotright}}1%
		{<<}{{\guillemotleft}}1%
		{«}{{\guillemotleft}}1,%
}%
% Zusätzliche Hervorherbungen (for every language)
\lstset{
	emph={%  
    	critical,
    	section
    },
    emphstyle={\color{red}\bfseries}%\underbar}%
}%
\lstdefinelanguage{C_var}
{
	language=C,
	morecomment=[l]{//},
	morekeywords={
		strlen,
	},
	morekeywords=[2]{
		malloc,
		recv,
		send,
		sem_init,
		sem_wait,
		sem_post,
		sem_getvalue,
		sem_destroy,
		raise,
		signal,
		pthread_create,
		pthread_join,
		pthread_detach,
		pthread_cancel,
		pthread_setcancelstate,
		pcap_breakloop,
		pcap_close
	},
	morekeywords=[3]{
		int8_t,
		uint8_t,
		int16_t,
		uint16_t,
		int32_t,
		uint32_t,
		int64_t,
		uint64_t,
		int128_t,
		uint128_t,
		uintptr_t,
		sem_t,
		pthread_t
	},
	morekeywords=[9]{
		critical section
	},
}
%Additional languages with keywords
\lstdefinelanguage{semaphore}
{
	morekeywords={
		type,
		record,
		end,
		if,
		then,
		else,
		loop
	},
	morekeywords=[2]{
		wait,
		signal,
		await,
		advance,
		sem_wait,
		sem_post,
		sem_init,
		sem_getvalue
	},
	morekeywords=[3]{
		val
	},
	morekeywords=[4]{
		ticket
	},
	sensitive=true,
	morecomment=[l]{//},
	morecomment=[s]{/*}{*/},
	morestring=[b]",
	morestring=[b]',
	keywordstyle=\bfseries\ttfamily\color{blue},
	keywordstyle=[2]\bfseries\ttfamily\color{teal},
	keywordstyle=[3]\bfseries\ttfamily\color{olive},
	keywordstyle=[4]\bfseries\ttfamily\color{olive},
	stringstyle=\color{green}\ttfamily,
}
%/\/\/\/\/\/\/\/\/\/\/\/\/\/\/\/\/\/\/\/\/\/\/\/\/\/\/\/\/\/\/\/\/\/\/\/\
%							Listings fertig
%------------------------------------------------------------------------
%________________________________________________________________________%%
\microtypesetup{activate=false}%
%%%%%%%%%%%%%%%%%%%%%%%%%%%%%%%%%%%%%%%%%%%%%%%%%%%%%%%%%%%%%%%%%
% Tracking, Spacing & Kerning werden seit Lua(La)Tex von 'fontspec' verwaltet
% 				% \microtypesetup{tracking=false}%
% 				% \microtypesetup{kerning=false}%
% 				% \microtypesetup{spacing=false}%
%
%================================================================
%----------------------------------------------------------------
%      Title Page \& Schmutztitel
%================================================================
%
%
% Standard-Title-Page Daten. Actually not used...
%
\title{}%
\author{DenKr}%
\date{\copyright \today}%
%
%
%
%
% Titlepage - Daten für eigene Title Page
%
\newcommand{\TPtitle}{%
Example\nl%
例\nl%
\begin{ocg}{Romaji TitlePage}{romaji-titlePage}{0}%
れい
rei%
\par\end{ocg}%
\switchocg{romaji-titlePage}{\fbox{\tiny ローマ字}}%
}%
\newcommand{\TPpublicationType}{\vspace{60pt}}%
\newcommand{\TPauthorAnrede}{}%
\newcommand{\TPauthorTitleBefore}{}%
\newcommand{\TPauthorTitleAfter}{M.Sc.}%
\newcommand{\TPauthorFirst}{Den}%
\newcommand{\TPauthorLast}{Kr}%
\newcommand{\TPauthorStreet}{}%
\newcommand{\TPauthorPLZ}{}%
\newcommand{\TPauthorCity}{}%
\newcommand{\TPauthorMail}{}%
\newcommand{\TPsignedAtCity}{}%
\newcommand{\TPsupervisorOneAnrede}{}%
\newcommand{\TPsupervisorOneTitleBefore}{}%
\newcommand{\TPsupervisorOneTitleAfter}{}%
\newcommand{\TPsupervisorOneFirst}{}%
\newcommand{\TPsupervisorOneLast}{}%
\newcommand{\TPsupervisorTwoAnrede}{}%
\newcommand{\TPsupervisorTwoTitleBefore}{}%
\newcommand{\TPsupervisorTwoTitleAfter}{}%
\newcommand{\TPsupervisorTwoFirst}{}%
\newcommand{\TPsupervisorTwoLast}{}%
\newcommand{\TPsupervisorThreeAnrede}{}%
\newcommand{\TPsupervisorThreeTitleBefore}{}%
\newcommand{\TPsupervisorThreeTitleAfter}{}%
\newcommand{\TPsupervisorThreeFirst}{A}%
\newcommand{\TPsupervisorThreeLast}{B}%
% \newcommand{\TPdate}{\today}%
\newcommand{\TPdate}{\today}%
%%
%\maketitle
%
%
%	Usage Notes:
% 	\ifdefempty needs the \usepackage{etoolbox}
%
%
%
% 
% 
% 
%
\newcommand{\HRule}{\rule{\linewidth}{0.5mm}}%
%
%
\begin{titlepage}%
%
\begin{center}%
%
%
% Oberer Teil der Titelseite:
\begin{minipage}{0.4\textwidth}%
%\includegraphics[height=0.15\textheight]{./bilder/logo.png}\\[1cm]
\begin{flushleft}%
% \includegraphics[height=4\baselineskip]{./graphics/core/Logos/Some_Logo.png}%
% \textsc{\Large Affiliation}\\
% \textsc{\Large }%
\end{flushleft}%
\end{minipage}%
\hfill%
\begin{minipage}{0.5\textwidth}%
\begin{flushright}%
% {\Large \fontfamily{phv}\fontseries{m}\fontshape{n}\selectfont Affil Part1}\\%
% {\Large \fontfamily{phv}\fontseries{m}\fontshape{n}\selectfont Affil Part2}\\%
% {\Large \fontfamily{phv}\fontseries{m}\fontshape{n}\selectfont Affil Part3}\\%
% {\Large \fontfamily{phv}\fontseries{m}\fontshape{n}\selectfont Affil Part4}%
\end{flushright}%
\end{minipage}%
%
\vspace{0.07\textheight}%
%
\begin{spacing}{1.2}%
\textsc{\LARGE\TPpublicationType}%
\end{spacing}%
%
\vspace{0.02\textheight}%
%
%
% Title
\HRule \\[0.2\baselineskip]%
\begin{spacing}{1}%
\huge\bfseries%
\TPtitle%
\\[0\baselineskip]%
\end{spacing}%
\HRule \\[0.08\textheight]%
%
% Author and supervisor
\begin{minipage}[t]{0.47\textwidth}%
\begin{flushleft} \large%
{Autor:}\\%
\ifdefempty{\TPauthorAnrede}%
	{}%
	{\TPauthorAnrede\ }%
\ifdefempty{\TPauthorTitleBefore}%
	{}%
	{\TPauthorTitleBefore\ }%
\TPauthorFirst\ \textsc{\TPauthorLast}%
\ifdefempty{\TPauthorTitleAfter}%
	{}%
	{, \TPauthorTitleAfter}%
\end{flushleft}%
\end{minipage}%
\hfill%
\begin{minipage}[t]{0.52\textwidth}%
\begin{flushright} \large%
% \vspace*{\baselineskip}%
% {Betreuer:}\\%
{\ }\\%
\ifdefempty{\TPsupervisorOneAnrede}%
	{}%
	{\TPsupervisorOneAnrede\ }%
\ifdefempty{\TPsupervisorOneTitleBefore}%
	{}%
	{\TPsupervisorOneTitleBefore\ }%
\TPsupervisorOneFirst\ \textsc{\TPsupervisorOneLast}%
\ifdefempty{\TPsupervisorOneTitleAfter}%
	{}%
	{, \TPsupervisorOneTitleAfter}%
\\%
\ifdefempty{\TPsupervisorTwoAnrede}%
	{}%
	{\TPsupervisorTwoAnrede\ }%
\ifdefempty{\TPsupervisorTwoTitleBefore}%
	{}%
	{\TPsupervisorTwoTitleBefore\ }%
\TPsupervisorTwoFirst\ \textsc{\TPsupervisorTwoLast}%
\ifdefempty{\TPsupervisorTwoTitleAfter}%
	{}%
	{, \TPsupervisorTwoTitleAfter}%
\\%
\end{flushright}%
\end{minipage}%
%
\vfill%
%
% Unterer Teil der Seite
{\large \TPdate}%
%
\end{center}%
%
\end{titlepage}%%
%
% Title Page Rückseite
%
\vspace*{\fill}%
\begin{flushright}%
\begin{large}%
\TPauthorFirst\ \TPauthorLast\nl%
% v0.1 - 02.09.2014\nl
% v1.0 - 03.11.2014\nl
% v2.0 - 28.01.2015\nl
% Final - 28.01.2015\nl
% aktuell: \today
\TPdate%
\end{large}%
\end{flushright}%
\clearpage%
%
%================================================================
%----------------------------------------------------------------
%      Inhaltsverzeichnis
%================================================================
%
\begingroup%
  \renewcommand*{\chapterpagestyle}{empty}%
  \pagestyle{plain}%
  \tableofcontents%
%   \thispagestyle{empty}%
  \clearpage%
%   \listoffigures%
%   \clearpage%
%   \listoftables%
%   \clearpage%
\endgroup%
%
%================================================================
%----------------------------------------------------------------
%      Setup, Typographisch
%             (Für eigentliches Dokument)
%================================================================
%
% \selectlanguage{ngerman}%
%
%\sloppy % Würde die Dehnung der Zwischenräume etwas lockerer machen
%\fussy % Schaltet \sloppy wieder aus. Ist der Latex-Standart
\microtypesetup{activate=true}%
%%%%%%%%%%%%%%%%%%%%%%%%%%%%%%%%%%%%%%%%%%%%%%%%%%%%%%%%%%%%%%%%%
% Tracking, Spacing & Kerning werden seit Lua(La)Tex von 'fontspec' verwaltet
% 					% \microtypesetup{tracking=true}%
% 					% \microtypesetup{kerning=true}%
% 					% \microtypesetup{spacing=true}%
%
% Beachte: Typographisch erhalten Kapitelseiten keinen Kolumnentitel: plain
\renewcommand*{\chapterpagestyle}{plain.scrheadings}%
\pagestyle{scrheadings}%
\renewcommand*{\partpagestyle}{empty}%
%
%
%
%
%
%________________________________________________________________________
%------------------------------------------------------------------------
%		Spacing für
%					- Equation Environment
%					- floats
%					- Some new Lengths & Variables
%/\/\/\/\/\/\/\/\/\/\/\/\/\/\/\/\/\/\/\/\/\/\/\/\/\/\/\/\/\/\/\/\/\/\/\/\
% - - - - - - - - - - - - - - - - - - - - - - - - - - - - - - - - - - - -
%- - - - - - - - - - - - - - - - - - - - - - - - - - - - - - - - - - - - -
% - - - - - - - - - - - - - - - - - - - - - - - - - - - - - - - - - - - -
%		Equation
% - - - - - - - - - - - - - - - - - - - - - - - - - - - - - - - - - - - -
%------------------------------------------------------------------------
% 	\begin{equation}\begin{split}
% 	MRes_{K}^{3rd} = \delta_{3rd} + \delta_{K}
% 	\end{split}\end{equation}
%------------------------------------------------------------------------
% Default for 'above' & 'below' Space:
%		11.0pt plus 3.0pt minus 6.0pt
% You can check this using (prints it into the Document):
%		\the\abovedisplayskip
%		\the\belowdisplayskip
%------------------------------------------------------------------------
% The ones with 'short' come in play if the last line immediately before an equation is, well, short.
%/\/\/\/\/\/\/\/\/\/\/\/\/\/\/\/\/\/\/\/\/\/\/\/\/\/\/\/\/\/\/\/\/\/\/\/\
\setlength{\abovedisplayskip}{3pt}
\setlength{\belowdisplayskip}{3pt}
\setlength{\abovedisplayshortskip}{0pt}
\setlength{\belowdisplayshortskip}{0pt}
% - - - - - - - - - - - - - - - - - - - - - - - - - - - - - - - - - - - -
%- - - - - - - - - - - - - - - - - - - - - - - - - - - - - - - - - - - - -
% - - - - - - - - - - - - - - - - - - - - - - - - - - - - - - - - - - - -
%		Floats
% - - - - - - - - - - - - - - - - - - - - - - - - - - - - - - - - - - - -
\setlength\extrarowheight{1ex}%
% The Spacing between floats and the surrounding text at top and bottom of the
% float. The Defaults are somewhat complicated and depend on document type and
% font-size%
% \setlength{\intextsep}{\intextsep}%
% Some more lenghts like this
%     \textfloatsep — distance between floats on the top or the bottom and the text;
%     \floatsep — distance between two floats;
%     \intextsep — distance between floats inserted inside the page text (using h) and the text proper.
% - - - - - - - - - - - - - - - - - - - - - - - - - - - - - - - - - - - -
%- - - - - - - - - - - - - - - - - - - - - - - - - - - - - - - - - - - - -
% - - - - - - - - - - - - - - - - - - - - - - - - - - - - - - - - - - - -
%		New Lenghts & Variables
% - - - - - - - - - - - - - - - - - - - - - - - - - - - - - - - - - - - -
\newlength\textheighttemp%
% \setlength{\textheighttemp}{\textheight}%
\newlength\textwidthtemp%
% \setlength{\textwidthtemp}{\textwidth}%
\newlength\textheightstd%
\setlength{\textheightstd}{\textheight}%
\newlength\textwidthstd%
\setlength{\textwidthstd}{\textwidth}%
\newlength\textheightold%
\newlength\textwidthold%
%
\newlength\tempheight%
\newlength\tempwidth%
%\/\/\/\/\/\/\/\/\/\/\/\/\/\/\/\/\/\/\/\/\/\/\/\/\/\/\/\/\/\/\/\/\/\/\/\/
%			Spacing End
%------------------------------------------------------------------------
%________________________________________________________________________
%
% ( '\setlength' must come after \begin{document} )
%/\/\/\/\/\/\/\/\/\/\/\/\/\/\/\/\/\/\/\/\/\/\/\/\/\/\/\/\/\/\/\/\/\/\/\/\
%			Spacing End
%------------------------------------------------------------------------
%________________________________________________________________________
%
%
%
%
%
%
%
%
%
%________________________________________________________________________
%------------------------------------------------------------------------
%							Setups für Microtype
%/\/\/\/\/\/\/\/\/\/\/\/\/\/\/\/\/\/\/\/\/\/\/\/\/\/\/\/\/\/\/\/\/\/\/\/\
% \SetProtrusion{encoding={*},family={bch},series={*},size={6,7}}
%               {1={ ,750},2={ ,500},3={ ,500},4={ ,500},5={ ,500},
%                6={ ,500},7={ ,600},8={ ,500},9={ ,500},0={ ,500}}
%%%%%%%%%%%%%%%%%%%%%%%%%%%%%%%%%%%%%%%%%%%%%%%%%%%%%%%%%%%%%%%%%
% Tracking, Spacing & Kerning werden seit Lua(La)Tex von 'fontspec' verwaltet
% 				% \SetExtraKerning[unit=space]
% 				%     {encoding={*}, family={bch}, series={*}, size={footnotesize,small,normalsize}}
% 				%     {\textendash={400,400}, % en-dash, add more space around it
% 				%      "28={ ,150}, % left bracket, add space from right
% 				%      "29={150, }, % right bracket, add space from left
% 				%      \textquotedblleft={ ,150}, % left quotation mark, space from right
% 				%      \textquotedblright={150, }} % right quotation mark, space from left
% 				% \SetTracking{encoding={*}, shape=sc}{40}
%\/\/\/\/\/\/\/\/\/\/\/\/\/\/\/\/\/\/\/\/\/\/\/\/\/\/\/\/\/\/\/\/\/\/\/\/
%							Microtype fertig
%------------------------------------------------------------------------
%________________________________________________________________________
%
%
%
%
%
%
%
%________________________________________________________________________
%------------------------------------------------------------------------
%							Setups für Glossaries
%/\/\/\/\/\/\/\/\/\/\/\/\/\/\/\/\/\/\/\/\/\/\/\/\/\/\/\/\/\/\/\/\/\/\/\/\
%Formatierung für Abkürzungen bei erstem Auftreten und weiteren und
%Falls deutsch Key im GLossar vorhanden, benutze ihn beim ersten Mal
% \defglsentryfmt[\acronymtype]{
%   \ifglsused{\glslabel}
%     {\glsgenentryfmt}% Wenn verwendet normales format
%     {\textit{\glsgenentryfmt}}% beim ersten mal kursiv
% }
% 		\newcommand*{\glsprintmitdeutsch}{%
% 			{\glsentrylong{\glslabel} (\glsentryname{\glslabel}, dt.:
% 			\glsentrydeutsch{gls-\glslabel})}%
% 		}%
% 		\newcommand*{\glsprintmitenglisch}{%
% 			{\glsentrylong{\glslabel} (\glsentryname{\glslabel}, engl.:
% 			\glsentryenglisch{gls-\glslabel})}%
% 		}%
% 		\newcommand*{\glsprintmitdualentry}{%
% 		  \ifglsused{\glslabel}%
% 		    {%
% 		    	\glsgenentryfmt%
% 		    }{%
% 		    	\ifglshasfield{deutsch}{gls-\glslabel}{%
% 		    		\glsprintmitdeutsch%
% 		    	}{%
% 			    	\ifglshasfield{englisch}{gls-\glslabel}{%
% 			    		\glsprintmitenglisch%
% 			    	}{%
% 			    		{\glsgenentryfmt}%
% 			    	}%
% 		    	}%
% 		    }%
% 		}%
% 		\newcommand*{\glsprintohnedeutsch}{%
% 		  \ifglsused{\glslabel}%
% 		    {\glsgenentryfmt}% Wenn verwendet normales format
% 		    {\glsgenentryfmt}% beim ersten mal anders formatiert
% 		}%
% 		% Alternativen für erste Formatierung:
% 		% textsf, textit, texttt
% 		%
% 		\defglsentryfmt[\acronymtype]{%
% 			\ifglsentryexists{gls-\glslabel}{%
% 				\glsprintmitdualentry%
% 			}{%
% 				\glsprintohnedeutsch%
% 			}%
% 		}%
% 		%
% 		\renewcommand*{\glsclearpage}{}%
% 		%
% 		\newlist{myglossarylist}{description}{10}%
% 		\setlist[myglossarylist]{%
% 			align=left,%
% 		% 	itemindent=0em,%
% 		% 	labelindent=0em,%
% 		% 	listparindent=0em,%
% 		}%
% 		\setlist[myglossarylist,2]{%
% 			align=left,%
% 		 	leftmargin=2em,%
% 			itemindent=-2em,%
% 		}%
% 		\newglossarystyle{list+url}
% 		{% based on list style (adapt as required)
% 		  \setglossarystyle{list}%
% 		    \renewcommand{\glossentry}[2]{%
% 		    \item[\glsentryitem{##1}%
% 		          \glstarget{##1}{\glossentryname{##1}}]
% 		       \glossentrydesc{##1}\glspostdescription\space##2%
% 		    \ifglshasfield{url}{##1}{\glspar
% 		     \glsletentryfield{\thisurl}{##1}{url}%
% 		     \expandafter\url\expandafter{\thisurl}}{}}%
% 		}%
% 		\newglossarystyle{list+Deutsch+Untertitel+alternativ+url}{%
% 			\renewenvironment{theglossary}%
% 				{\begin{myglossarylist}}{\end{myglossarylist}}%
% 			\renewcommand*{\glossaryheader}{}%
% 			\renewcommand*{\glsgroupheading}[1]{}%
% 			\renewcommand*{\glossentry}[2]{%
% 				\item[\glsentryitem{##1}%
% 					  \glstarget{##1}{\glossentryname{##1}}]%
% 		%
% 				\ifglshasfield{untertitel}{##1}{\normalfont%
% 					\textbf{(}\glsentryuntertitel{##1}\textbf{)}.\space}{}%
% 				\ifglshasfield{alternativ}{##1}{\normalfont%
% 					Auch genannt: \textbf{\glsentryalternativ{##1}}.\space}{}%
% 				\ifglshasfield{deutsch}{##1}{Zu Deutsch: %
% 					\textbf{\glsentrydeutsch{##1}}.\space}{}%
% 				\glossentrydesc{##1}\glspostdescription\space ##2%
% 		    \ifglshasfield{url}{##1}{\glspar
% 				\url{\glsentryurl{##1}}}{}%
% 			}%
% 			\renewcommand*{\subglossentry}[3]{%
% 				\begin{myglossarylist}%
% 					\item[\glssubentryitem{##2}%
% 					\glstarget{##2}{\strut\glossentryname{##2}%
% 					}]%
% 		%
% 					\ifglshasfield{untertitel}{##2}{\normalfont%
% 						\textbf{(}\glsentryuntertitel{##2}\textbf{)}.\space}{}%
% 					\ifglshasfield{alternativ}{##2}{\normalfont%
% 						Auch genannt: \textbf{\glsentryalternativ{##2}}.\space}{}%
% 					\ifglshasfield{deutsch}{##2}{Zu Deutsch: %
% 						\textbf{\glsentrydeutsch{##2}}.\space}{}%
% 					\glossentrydesc{##2}\glspostdescription\space ##3.%
% 				\end{myglossarylist}%
% 			}%
% 			\renewcommand*{\glsgroupskip}{\ifglsnogroupskip\else\indexspace\fi}%
% 		}%
% 		\setglossarystyle{list}%
%
% \setacronymstyle{long-short}
%/\/\/\/\/\/\/\/\/\/\/\/\/\/\/\/\/\/\/\/\/\/\/\/\/\/\/\/\/\/\/\/\/\/\/\/\
%							Glossaries fertig
%------------------------------------------------------------------------
%________________________________________________________________________
%
%
%
%
%________________________________________________________________________
%------------------------------------------------------------------------
%							Setups für enumitem
%					Listen - (itemize, enumerate, description)
%/\/\/\/\/\/\/\/\/\/\/\/\/\/\/\/\/\/\/\/\/\/\/\/\/\/\/\/\/\/\/\/\/\/\/\/\
% \setlist[1]{\labelindent=\parindent} % < Usually a good idea
\setlist[description]{font=\sffamily\bfseries} % the Default
%/\/\/\/\/\/\/\/\/\/\/\/\/\/\/\/\/\/\/\/\/\/\/\/\/\/\/\/\/\/\/\/\/\/\/\/\
%							enumitem fertig
%------------------------------------------------------------------------
%________________________________________________________________________
%
%
%
%
%
%
%________________________________________________________________________
%------------------------------------------------------------------------
%						Setups für Hyperref/URL
%		Note: Package hyperref internally laods package url
%		(Like some other Packages do. For Example: biblatex)
%/\/\/\/\/\/\/\/\/\/\/\/\/\/\/\/\/\/\/\/\/\/\/\/\/\/\/\/\/\/\/\/\/\/\/\/\
\mathchardef\UrlBreakPenalty=100% I think default is 100
% Allow linebreaks in URLs after additional character to the defaults, but
% don't just \renew or \def it, because this would remove all characters
% predefined within \UrlBreaks in the package. This could mislead other users.
% E.g. can cause smaller dots.. So use the etoolbox with "append to"
\appto\UrlBreaks{%
	\do\*%
% 	\do\-% Is done with the url option "hyphens"
	\do\~%
	\do\"%
	\do\a\do\b\do\c\do\d\do\e\do\f\do\g\do\h\do\i\do\j%
	\do\k\do\l\do\m\do\n\do\o\do\p\do\q\do\r\do\s\do\t\do\u\do\v\do\w%
	\do\x\do\y\do\z%
}%
%	Note: Alread inside the \UrlBreaks by default are:
% 	\do\.%
% 	\do\=%
% 	\do\'%
% 	\do\&%
% 	\do\-% With the option hypens
% \expandafter\def\expandafter\UrlBreaks\expandafter{\UrlBreaks% save the current one
% 	\do\*%
% 	\do\-%
% 	\do\~%
% 	\do\'%
% 	\do\"%
% 	\do\-%
% 	\do\&%
% 	\do\=%
% }%
%/\/\/\/\/\/\/\/\/\/\/\/\/\/\/\/\/\/\/\/\/\/\/\/\/\/\/\/\/\/\/\/\/\/\/\/\
%						Hyperref/URL fertig
%------------------------------------------------------------------------
%________________________________________________________________________
%
%
%
%
%
%
%
%________________________________________________________________________
%------------------------------------------------------------------------
%						Redefinition of \texttt
%				to allow a proper hyphenation (Silbentrennung)
% other breakpoint symbols can be added, for example, below I also make [ a breakpoint:
%/\/\/\/\/\/\/\/\/\/\/\/\/\/\/\/\/\/\/\/\/\/\/\/\/\/\/\/\/\/\/\/\/\/\/\/\
\let\stdtexttt\texttt
% \newcommand{\newtexttt}[1]{%
%   \begingroup
%   \ttfamily
%   \begingroup\lccode`~=`/\lowercase{\endgroup\def~}{/\discretionary{}{}{}}%
%   \begingroup\lccode`~=`[\lowercase{\endgroup\def~}{[\discretionary{}{}{}}%
%   \begingroup\lccode`~=`.\lowercase{\endgroup\def~}{.\discretionary{}{}{}}%
%   \catcode`/=\active\catcode`[=\active\catcode`.=\active
%   \scantokens{#1\noexpand}%
%   \endgroup
% }
% Oh noes, better way: Look at the Schriftart Segment in the Header
%/\/\/\/\/\/\/\/\/\/\/\/\/\/\/\/\/\/\/\/\/\/\/\/\/\/\/\/\/\/\/\/\/\/\/\/\
%						Redef fertig
%------------------------------------------------------------------------
%________________________________________________________________________
%
%
%
%
%
%________________________________________________________________________
%------------------------------------------------------------------------
%							Setups für Cleveref
%/\/\/\/\/\/\/\/\/\/\/\/\/\/\/\/\/\/\/\/\/\/\/\/\/\/\/\/\/\/\/\/\/\/\/\/\
\crefname{chapter}{Kapitel}{Kapiteln}
\crefname{section}{Abschnitt}{Abschnitten}
\crefname{part}{Teil}{Teilen}
\crefname{figure}{Abbildung}{Abbildungen}
\crefname{table}{Tabelle}{Tabellen}
\crefname{listing}{Listing}{Listings}
\crefname{equation}{Formel}{Formeln}
%/\/\/\/\/\/\/\/\/\/\/\/\/\/\/\/\/\/\/\/\/\/\/\/\/\/\/\/\/\/\/\/\/\/\/\/\
%							Cleveref fertig
%------------------------------------------------------------------------
%________________________________________________________________________
%
%
%
%
%
%________________________________________________________________________
%------------------------------------------------------------------------
%						Definitionen allgemeiner Farben
%/\/\/\/\/\/\/\/\/\/\/\/\/\/\/\/\/\/\/\/\/\/\/\/\/\/\/\/\/\/\/\/\/\/\/\/\
\definecolor{optionalargscolor}{RGB}{255,125,25}%
\definecolor{ColorRegExCmdID}{RGB}{25,0,255}%
\definecolor{ColorRegExCmdSelect}{RGB}{255,125,25}%
\definecolor{ColorRegExCmdMultiSelect}{RGB}{30,150,0}%
\definecolor{ColorRegExCmdOption}{RGB}{100,100,100}%
%/\/\/\/\/\/\/\/\/\/\/\/\/\/\/\/\/\/\/\/\/\/\/\/\/\/\/\/\/\/\/\/\/\/\/\/\
%							Farben fertig
%------------------------------------------------------------------------
%________________________________________________________________________
%
%
%
%
%
%
%
%
% Initialisierungen wie Counter, Theorem, Variablen, Konstanten..
%
\theoremstyle{break}
\theoremheaderfont{\bfseries}
\theorembodyfont{\normalfont}
\theoremseparator{}
\theorempreskip{0ex}
\theorempostskip{0ex}
\theoremindent0.5em
%
%
% 
% 
%
%
%
%
%________________________________________________________________________
%------------------------------------------------------------------------
%							listings - Setup. Source-Code Einbindung.
%/\/\/\/\/\/\/\/\/\/\/\/\/\/\/\/\/\/\/\/\/\/\/\/\/\/\/\/\/\/\/\/\/\/\/\/\
\newcommand{\lstC}[1]{\lstinline[language=C,breaklines=true,basicstyle=\listingsfontinline]$#1$}
% Colors etc. to get the Eclipse Look
% \newfontfamily\listingsfont[Scale=0.7]{Courier} 
% \newfontfamily\listingsfontinline[Scale=0.8]{Courier New} 
\definecolor{eclipse_comment}{rgb}{0.12, 0.38, 0.18 } %adjusted, in Eclipse: {0.25, 0.42, 0.30 } = #3F6A4D
\definecolor{eclipse_keyword1}{RGB}{128, 0, 40}  % red
\definecolor{eclipse_keyword2}{RGB}{155,40,128} %purple
\definecolor{eclipse_keyword3}{RGB}{0,120,10} %green
\definecolor{eclipse_string}{rgb}{0.06, 0.10, 0.98} % #101AF9
\def\lstsmallmath{\leavevmode\ifmmode \scriptstyle \else  \fi}
\def\lstsmallmathend{\leavevmode\ifmmode  \else  \fi}
%-----------
\definecolor{lstbg}{RGB}{230,230,230}
\lstset{language=C,texcl=true}
% Basic lstset (optical settings)
% \lstset{
% 	numbers=left,
% 	numberstyle=\tiny,
% 	numbersep=5pt,
% 	tabsize=4,
% 	xleftmargin=0em,
% 	backgroundcolor=\color{lstbg},
% 	keywordstyle=\bfseries\ttfamily\color{blue},
% 	stringstyle=\color{green}\ttfamily,
% 	commentstyle=\color{gray}\ttfamily,
% }
% lstset for Look like Eclipse
\lstset{
	frame=shadowbox,
	numbers=left,
	numberstyle=\tiny,
	numbersep=5pt,
	showspaces=false,
	showtabs=false,
	tabsize=2,
	breaklines=true,
	keepspaces=true,
% 	basicstyle=\listingsfont,
	captionpos=b,
	xleftmargin=0em,
	xrightmargin=0.3em,
	backgroundcolor=\color{lstbg},
	keywordstyle=\color{eclipse_keyword1}\bfseries,
	keywordstyle=[2]\color{eclipse_keyword2}\bfseries,
	keywordstyle=[3]\color{eclipse_keyword3}\bfseries,
	keywordstyle=[9]\color{red}\bfseries,
	stringstyle=\color{eclipse_string},
	commentstyle=\color{eclipse_comment}\itshape,
	escapebegin={\lstsmallmath}, escapeend={\lstsmallmathend}
}
\lstset{%
	inputencoding=utf8,%
	extendedchars=true,%
	literate=%
		{ß}{{\ss}}2%
		{ö}{{\"o}}1%
		{ä}{{\"a}}1%
		{ü}{{\"u}}1%
		{Ö}{{\"O}}1%
		{Ä}{{\"A}}1%
		{Ü}{{\"U}}1%
		{>>}{{\guillemotright}}1%
		{»}{{\guillemotright}}1%
		{<<}{{\guillemotleft}}1%
		{«}{{\guillemotleft}}1,%
}%
% Zusätzliche Hervorherbungen (for every language)
\lstset{
	emph={%  
    	critical,
    	section
    },
    emphstyle={\color{red}\bfseries}%\underbar}%
}%
\lstdefinelanguage{C_var}
{
	language=C,
	morecomment=[l]{//},
	morekeywords={
		strlen,
	},
	morekeywords=[2]{
		malloc,
		recv,
		send,
		sem_init,
		sem_wait,
		sem_post,
		sem_getvalue,
		sem_destroy,
		raise,
		signal,
		pthread_create,
		pthread_join,
		pthread_detach,
		pthread_cancel,
		pthread_setcancelstate,
		pcap_breakloop,
		pcap_close
	},
	morekeywords=[3]{
		int8_t,
		uint8_t,
		int16_t,
		uint16_t,
		int32_t,
		uint32_t,
		int64_t,
		uint64_t,
		int128_t,
		uint128_t,
		uintptr_t,
		sem_t,
		pthread_t
	},
	morekeywords=[9]{
		critical section
	},
}
%Additional languages with keywords
\lstdefinelanguage{semaphore}
{
	morekeywords={
		type,
		record,
		end,
		if,
		then,
		else,
		loop
	},
	morekeywords=[2]{
		wait,
		signal,
		await,
		advance,
		sem_wait,
		sem_post,
		sem_init,
		sem_getvalue
	},
	morekeywords=[3]{
		val
	},
	morekeywords=[4]{
		ticket
	},
	sensitive=true,
	morecomment=[l]{//},
	morecomment=[s]{/*}{*/},
	morestring=[b]",
	morestring=[b]',
	keywordstyle=\bfseries\ttfamily\color{blue},
	keywordstyle=[2]\bfseries\ttfamily\color{teal},
	keywordstyle=[3]\bfseries\ttfamily\color{olive},
	keywordstyle=[4]\bfseries\ttfamily\color{olive},
	stringstyle=\color{green}\ttfamily,
}
%/\/\/\/\/\/\/\/\/\/\/\/\/\/\/\/\/\/\/\/\/\/\/\/\/\/\/\/\/\/\/\/\/\/\/\/\
%							Listings fertig
%------------------------------------------------------------------------
%________________________________________________________________________%