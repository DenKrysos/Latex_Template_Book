\documentclass[%
	fontsize=11pt,%
	paper=A4,%
	twoside=true,%
	titlepage,%
	openany,%
	numbers=noenddot,%
	DIV=calc,%
	headings=optiontoheadandtoc,% Damit wird im Kopf der optionale Kapitelname verwendet
% 	draft,%
]{scrbook}%
%[2014/09/13]%
%
% \usepackage{etex}%
% \reserveinserts{32}%
%
% Das Prozent-Zeichen leitet einen Kommentar ein,
% es hilft ebenso, im Text Leerzeichen zu unterbinden.
%
% fontsize=12pt  Schriftgroesse in 10, 11 oder 12 Punkt
% a4paper        Papierformat ist hier A4
% landscape      Querformat wird natürlich unterstützt ;-)
% parskip        Absatzabstand anstatt Einzüge
% draft          Der Entwurfsmodus deckt Schwächen auf
%			e.g. Picutres erscheinen nur als Umrissboxen mit Bildname
% {scrartcl}     Die Dokumentenklasse book, report, article
%                oder fürs deutsche scrbook, scrreprt, scrartcl
%
% bibtex and biber are external programs that process bibliography information
% and act (roughly) as the interface between your .bib file and your LaTeX document.
% -> To prefer: biber
% natbib and biblatex are LaTeX packages that format citations and
% bibliographies; natbib works only with bibtex, while biblatex (at the moment)
% works with both bibtex and biber.)
%
%
%
% INFO:
% \paperheight = 
%				\topmargin +
%				\headheight +
%				\headsep +
%				\textheight +
%				\footskip +
%				\footheight +
%				\ %Some more Space at the bottom
%
% \paperwidth = 
%				\oddsidemargin +
%				\textwidth +
%				\marginparsep +
%				\marginparwidth +
%				\ %Some more Space at the "left"
%
%
%
%________________________________________________________________________
%------------------------------------------------------------------------
%							PassOptions Preparation
%	Die PassOptions sollten vor der ersten Inklusion des Packages erfolgen
%	Also auch bevor ein Package von einem anderen eingebunden wird...
%/\/\/\/\/\/\/\/\/\/\/\/\/\/\/\/\/\/\/\/\/\/\/\/\/\/\/\/\/\/\/\/\/\/\/\/\
\PassOptionsToPackage{usenames,dvipsnames,svgnames,x11names,table,prologue}{xcolor}%
\PassOptionsToPackage{hyphens}{url}%
\PassOptionsToPackage{nomessages}{fp}%
%/\/\/\/\/\/\/\/\/\/\/\/\/\/\/\/\/\/\/\/\/\/\/\/\/\/\/\/\/\/\/\/\/\/\/\/\
%							PassOptions fertig
%------------------------------------------------------------------------
%========================================================================
%
%
%
%
% Package Includes - maybe some immediate with options
%
%________________________________________________________________________
%------------------------------------------------------------------------
%					Basic Language depending Packages
%/\/\/\/\/\/\/\/\/\/\/\/\/\/\/\/\/\/\/\/\/\/\/\/\/\/\/\/\/\/\/\/\/\/\/\/\
% \usepackage[english,german,ngerman]{babel}%
\usepackage[babelshorthands]{polyglossia}%
\setmainlanguage[spelling=new]{german}% Synonym: \setdefaultlanguage
\setotherlanguage[variant=british]{english}
%\selectlanguage{japanese}%
% - - - - - - - - - - - - - - - - - - - - - - - - - - - - - - - - - - -
% \renewcommand\thesection{\arabic{section}}
% \renewcommand\thefigure{\arabic{section}.\arabic{figure}}
\renewcommand\theequation{\arabic{chapter}.\arabic{equation}}%
% - - - - - - - - - - - - - - - - - - - - - - - - - - - - - - - - - - -
% == Note ==  With csquotes.sty, use
%     \textquote{}, to automatically print correct quotation marks, depending on the set language by polyglossia (or babel)
%     \enquote{}, for 'nested' quotations
\usepackage[autostyle=true,german=quotes]{csquotes}% Deutsche Anführungszeichen, german=quotes, guillemets, swiss
% \usepackage[babel,style=english]{csquotes}%
% - - - - - - - - - - - - - -
% == Info, »ALT-Codes« == Some specific quotation signs for hardcoding, using UTF-8 encoding. How-To-Input, if you like.
%    ALT(left) + [Sequence on NumPad]  (On Windows) (Using ASCII Decimal Index)
%      "  -    34  (Same as SHIFT + 2, Quotation Mark)
%      »  -   175  (Guillemets, Chevrons. German: Open. Swiss,French: Close)
%      «  -   174  (Guillemets, Chevrons. German: Close. Swiss,French: Open)
%      ›  -  0155
%      ‹  -  0139
%      „  -  0132  (low curly doublequote)
%      “  -  0147  (curly double open quote)
%      ”  -  0148  (curly double close quote)
%      ‘  -  0145  (curly single open quote)
%      ’  -  0146  (curly single close quote)
%      ‚  -  0130  (curly single quote)
% - - - - - - - - - - - - - -
%/\/\/\/\/\/\/\/\/\/\/\/\/\/\/\/\/\/\/\/\/\/\/\/\/\/\/\/\/\/\/\/\/\/\/\/\
%							END Basic Language
%------------------------------------------------------------------------
%________________________________________________________________________
%
%
%
%
%
%
%________________________________________________________________________
%------------------------------------------------------------------------
%							Schriftarten einstellen
%/\/\/\/\/\/\/\/\/\/\/\/\/\/\/\/\/\/\/\/\/\/\/\/\/\/\/\/\/\/\/\/\/\/\/\/\
% 								%##########################
% 								% Old `fontenc'-Stuff, pre Lua(La)Tex
% 								%##########################
% 								%|=======================||||
% 								% \usepackage[T1]{fontenc}%|| Alt, pre-Lua(La)Tex
% 								%|=======================||||
% 								% % %\usepackage[scaled]{uarial}
% 								% % % \usepackage{bookman}
% 								%|===================||||
% 								% \usepackage{lmodern}%|| Lädt Latin Modern Font und setzt es fürs Dokument
% 								%|===================||||
% 								% % %Ist weniger Pixelig in pdf als Latex Standart
% 								%|====================||||
% 								% \usepackage{textcomp}%|| Lädt Text Companion Font
% 								%|====================||||
% 								% % %Stellt insbesondere Zeichen zur verfügung, wie
% 								% % %baht, bul­let, copy­right, mu­si­cal­note, onequar­ter, sec­tion, and yen
% 								% % %------------------------------------------------
% 								% % % Standartschriftart festlegen:
% 								% % % Mögliche Werte
% 								% % % \rmdefault - Roman (Serifen) Font
% 								% % % \sfdefault - Sans Serif Font
% 								% % % \ttdefault - TypeWriter Font
% 								%|=========================================||||
% 								% \renewcommand*{\familydefault}{\rmdefault}%||
% 								%|=========================================||||
% 								% % %Die drei Schriftfamilien einstellen:
% 								%|==============================||||
% 								% \renewcommand*{\rmdefault}{lmr}%||
% 								%|==============================||||
% 								% % % Standartmäßig verfügbare Schriftarten:
% 								% % % cmr 	Computer Modern Roman (default)
% 								% % % lmr 	Latin Modern Roman
% 								% % % pbk 	Bookman
% 								% % % bch 	Charter
% 								% % % pnc 	New Century Schoolbook
% 								% % % ppl 	Palatino
% 								% % % ptm 	Times
% 								%|===============================||||
% 								% \renewcommand*{\sfdefault}{lmss}%||
% 								%|===============================||||
% 								% % % Standartmäßig verfügbare Schriftarten:
% 								% % % cmss 	Computer Modern Sans Serif (default)
% 								% % % lmss 	Latin Modern Sans Serif
% 								% % % pag 	Avant Garde
% 								% % % phv 	Helvetica
% 								%|===============================||||
% 								% \renewcommand*{\ttdefault}{lmtt}%||
% 								%|===============================||||
%####################################
% Now: 'fontspec'
%####################################
\usepackage{fontspec}%
% - - - - Fonts - - - 
%  --  --  The lower, the better i like them
% \setmainfont{Times New Roman}%
% \setmainfont{Gungsuh}%
% \setmainfont{Malgun}%
% \setmainfont{Batang}%
% \setmainfont{Meiryo}%
% \setmainfont{MS Mincho}%
% \setmainfont{MS Gothic}%
% \setmainfont{NSimSun}%
% \setmainfont{SimSun}%
% \setmainfont{Kozuka Gothic Pro}% From Adobe, Downloadable free at: http://fontpark.net/de/schriftart/kozuka-gothic-pro-b/
% \setmainfont{Meiryo UI}%
% \setmainfont{Yu Gothic}%
%		\setmainfont{Yu Gothic Light}%
%		\setmainfont{Yu Gothic Medium}%
% \setmainfont{Yu Gothic UI}%
%		\setmainfont{Yu Gothic UI Light}%
%		\setmainfont{Yu Gothic UI Semibold}%
%		\setmainfont{Yu Gothic UI Semilight}%
% - - - - - - - - - - - - - - - - - - - - - - - - - - - - - - -
% \setmainfont[ItalicFont={Malgun},Ligatures=TeX]{Yu Gothic UI}% Includes Japanese Support, but not separated.
% 		Like:	% 	\setmainjfont{Yu Gothic UI} % \mcfamily
% 		 +		% 	\setsansjfont{Malgun} % \gtfamily
\setmainfont{Times New Roman}%
% - - - - - - - - - - - - - - - - - - - - - - - - - - - - - - -
% For Japanese Characters
\usepackage{luatexja}
\usepackage{luatexja-fontspec}
% \setmainjfont{Yu Gothic UI}% \mcfamily
\setmainjfont{Meiryo UI}% \mcfamily
\setsansjfont{Malgun}% \gtfamily
%<-------------------->
% % % Create a command to format as typewrite with enabled hyphenation
% % % (Silbentrennung)
%<-------------------->
\makeatletter
\DeclareRobustCommand\ttfamily
        {\not@math@alphabet\ttfamily\mathtt
         \fontfamily\ttdefault\selectfont\hyphenchar\font=-1\relax}
\makeatother
\DeclareTextFontCommand{\texttthyph}{\ttfamily\hyphenchar\font=45\relax}
% % % Standartmäßig verfügbare Schriftarten:
% % % cmtt 	Computer Modern Typewriter (default)
% % % lmtt 	Latin Modern
% % % pcr 	Courier
%/\/\/\/\/\/\/\/\/\/\/\/\/\/\/\/\/\/\/\/\/\/\/\/\/\/\/\/\/\/\/\/\/\/\/\/\
%							Schriftarten fertig
%------------------------------------------------------------------------
%________________________________________________________________________
%
%
%
%
%
%
%________________________________________________________________________
%------------------------------------------------------------------------
%					Some 'Specials', one might want/like
%/\/\/\/\/\/\/\/\/\/\/\/\/\/\/\/\/\/\/\/\/\/\/\/\/\/\/\/\/\/\/\/\/\/\/\/\
%________________________________________________________________________
%------------------------------------------------------------------------
\KOMAoptions{%
% 	BCOR=8mm,% Bindeverlust von 8mm am Innenrand einbeziehen
	DIV=last,%
}%
\usepackage[%
 	right=1.5cm,% Entspricht im Falle von twoside dem Außenrand
  	left=1cm,% Entspricht im Falle von twoside dem Innennrand
 	top=2.5cm,%
  	bottom=2.5cm,%
]{geometry}%manuelles Anpassen der Seitenränder
\setlength{\marginparwidth}{2.5cm}%Anpassung der Breite der Randnotiz.
% Zu Beachten, wenn Seiten manuell gewählt wird.
\usepackage{marginnote}%
\usepackage{marginfix}%
\renewcommand*{\marginfont}{\footnotesize}%
\usepackage{hyphsubst}%
\usepackage{todonotes}%
\usepackage{setspace}%
%
%
%
% See PassOptions for xcolor on Top
\usepackage{xcolor}%
\definecolor{citegreen}{RGB}{0,180,0}%
\definecolor{todonotecol}{RGB}{250,0,0}%
\usepackage[ngerman]{varioref}%
%
%
%
%
%For \DeclareDocumentCommand (bessere Makro-Definition als \newcommand) and so on
\usepackage{xparse}
%	Also Necessary to create the DualEntry Makro for Glossaries
%
%
% 'hyperref' is loaded further below. It is to prefer, to load 'hyperref' as late as possible.
%
%________________________________________________________________________
%------------------------------------------------------------------------
%							Für Kopf und Fußzeilen der Seiten
%							Alles auf KOMA basierend
%/\/\/\/\/\/\/\/\/\/\/\/\/\/\/\/\/\/\/\/\/\/\/\/\/\/\/\/\/\/\/\/\/\/\/\/\
% \usepackage{scrpage2} %Älter
\usepackage{scrlayer-scrpage}%
\PreventPackageFromLoading{fancyhdr}%
%/\/\/\/\/\/\/\/\/\/\/\/\/\/\/\/\/\/\/\/\/\/\/\/\/\/\/\/\/\/\/\/\/\/\/\/\
%							Kopf- Fußzeilen fertig
%------------------------------------------------------------------------
%________________________________________________________________________
%/\/\/\/\/\/\/\/\/\/\/\/\/\/\/\/\/\/\/\/\/\/\/\/\/\/\/\/\/\/\/\/\/\/\/\/\
%							END 'Specials'
%------------------------------------------------------------------------
%________________________________________________________________________
%
%
%
%
%
%
%________________________________________________________________________
%------------------------------------------------------------------------
%					TikZ
%/\/\/\/\/\/\/\/\/\/\/\/\/\/\/\/\/\/\/\/\/\/\/\/\/\/\/\/\/\/\/\/\/\/\/\/\
% 	!!! IMPORTANT !!!
% 	Never forget -shell-escape
% 	as argument for pdflatex
% 	when using standalone
\newcommand{\includestandalonedefaultmode}{buildnew}%
\newcommand{\tikzFilesPath}{./graphics/tikz/}%
\usepackage[%
	group=true,%
	mode=\includestandalonedefaultmode,%
% 	subpreambles=true,%
% 	build={%
% 		latexoptions=-interaction=batchmode -shell-escape -jobname ’\buildjobname’%
% 	},%
	]{standalone}%
\usepackage{tikz}%
\usetikzlibrary{calc}%
\usetikzlibrary{matrix}%
\usetikzlibrary{fit}%
\usetikzlibrary{positioning}%
\usetikzlibrary{shapes}%
\usetikzlibrary{shapes.geometric}%
\usetikzlibrary{decorations}%
\usetikzlibrary{decorations.markings}%
\usetikzlibrary{decorations.pathmorphing}% 
\usetikzlibrary{arrows}%
\usetikzlibrary{arrows.meta}%
\usepackage{pgfplots}%
\pgfplotsset{compat=1.11}%
\usepgfplotslibrary{dateplot}%
\usetikzlibrary{shadows}%
\usetikzlibrary{shadows.blur}%
\usetikzlibrary{fadings}%
\usetikzlibrary{shadings}%
%/\/\/\/\/\/\/\/\/\/\/\/\/\/\/\/\/\/\/\/\/\/\/\/\/\/\/\/\/\/\/\/\/\/\/\/\
%					TikZ End
%------------------------------------------------------------------------
%________________________________________________________________________
%
%
%
%
%
%
%________________________________________________________________________
%------------------------------------------------------------------------
%					About Tables
%/\/\/\/\/\/\/\/\/\/\/\/\/\/\/\/\/\/\/\/\/\/\/\/\/\/\/\/\/\/\/\/\/\/\/\/\
%
%%%%\usepackage{tabular}% Latex-Std, just don't use
%\usepackage{tabular*}% Latex-Std, bit better
%%%%\usepackage{tabularx}% More comfortable than, 'tabular*', better to use, but no nesting
\usepackage{tabulary}% Nearly the same as 'tabularx', bit more extensive
\usepackage{makecell}% http://ctan.org/pkg/makecell
%------------------------------------------------------------------------------------
% -- Table recommendation:
% ---- Use 'tabulary', together with 'makecell' (if necessary)
% ---- -- Could also use 'tabularx', together with 'makecell', if you like it better...
% ---- Otherwise maybe 'tabular*'
%------------------------------------------------------------------------------------
%
% Nice for use with e.g. Tables / Legends.
%     Look into the Folder 'organization/templates' for a Table Example inclusive Legend
\definecolor{geilesrot}{rgb}{ .753,  0,  0}
\definecolor{angenehmesorange}{rgb}{ 1,  .753,  0}
\definecolor{superduftesgruen}{rgb}{ 0,  .69,  .314}
\definecolor{woDieSonneNieHinscheintSchwarz}{rgb}{ 0,  0,  0}
%----------------------------------------------------------------
	\newcommand{\legcrossstraight}{\textcolor{geilesrot}{$\times$}}
\newcommand{\legcross}{\textbf{\huge\legcrossstraight}}
	\newcommand{\legbulletstraight}{\textcolor{angenehmesorange}{$\bullet$}}
\newcommand{\legbullet}{\textbf{\huge\legbulletstraight}}
	\newcommand{\legcheckstraight}{\textcolor{superduftesgruen}{$\checkmark$}}
\newcommand{\legcheck}{\textbf{\huge\legcheckstraight}}
%/\/\/\/\/\/\/\/\/\/\/\/\/\/\/\/\/\/\/\/\/\/\/\/\/\/\/\/\/\/\/\/\/\/\/\/\
%					Tables End
%------------------------------------------------------------------------
%________________________________________________________________________
%
%
%
%
%
%
%________________________________________________________________________
%------------------------------------------------------------------------
%					Various (Basic) Packages
%/\/\/\/\/\/\/\/\/\/\/\/\/\/\/\/\/\/\/\/\/\/\/\/\/\/\/\/\/\/\/\/\/\/\/\/\

%- - - - - - - - - - - - - - - - - - - - - - - - - - - - - - -
% *** SUBFIGURE PACKAGES ***  % As Note: subfigure is dead, rather use subfig
%	Add the option 'subfigure' to 'tocloft' when using both concurrently
% \ifCLASSOPTIONcompsoc
   \usepackage[caption=false,font=normalsize,labelfont=sf,textfont=sf]{subfig}
% \else
%    \usepackage[caption=false,font=footnotesize]{subfig}
% \fi
%- - - - - - - - - - - - - - - - - - - - - - - - - - - - - - -
%
\usepackage{framed}%
\usepackage{float}%
\usepackage{wrapfig}%
\usepackage{varwidth}%
\usepackage[subfigure]{tocloft}%Ermöglicht customization von Inhaltsverzeichnis,
%					 Abbildungsverzeichnis und Tabellenverzeichnis
%
%
% Für Source-Code Hervorhebung/Formatierung
% package listings:
% scheint wohl "minderwertig" zu sein, jedoch tex-only solution
% \usepackage{tabularx}
\usepackage{listings}%
\usepackage{listingsutf8}%
%\usepackage{minted}%braucht externes Programm
%
\usepackage{xkeyval}%Allows to extend Package's Key-Sets with prefixes
% \usepackage{amsmath}%
\usepackage{mathtools}% Supersedes and internally loads 'amsmath'
\usepackage{amssymb}% Loads 'amsfonts'
\usepackage{ntheorem}%
\usepackage{calc}%
\usepackage{ifthen}%
\usepackage{enumitem}%
\usepackage{ulem}\normalem%
% \usepackage{bigfoot}%
\usepackage{caption}%
% \usepackage{subcaption}
\usepackage{graphicx}%
\usepackage{relsize}%
\usepackage{scrhack}%
\usepackage{diagbox}%
\usepackage{multirow}%
\usepackage{colortbl}%
\usepackage{printlen}%
\usepackage{xspace}%
\usepackage{units}%contains package 'nicefrac'. Supplies some nice Fraction formatting.
\usepackage{etoolbox}%
\usepackage{xstring}%
\uselengthunit{cm}%
% \usepackage{ragged2e}%
% \usepackage[htt]{hyphenat}%
% \usepackage{import}%
%-----------------------------------------------------------------------------------
% Microtype
\usepackage[%
	activate={true,%
	nocompatibility},%
	final,%
%%%%%%%%%%%%%%%%%%%%%%%%%%%%%%%%%%%%%%%%%%%%%%%%%%%%%%%%%%%%%%%%%
% Tracking, Spacing & Kerning werden seit Lua(La)Tex von 'fontspec' verwaltet
% 	tracking=false,%
% 	kerning=false,%
% 	spacing=false,%
	factor=1100,%
	stretch=10,%
	shrink=10%
]{microtype}%
% activate={true,nocompatibility} - activate protrusion and expansion
% final - enable microtype; use "draft" to disable
% tracking=true, kerning=true, spacing=true - activate these techniques
% factor=1100 - add 10% to the protrusion amount (default is 1000)
% stretch=10, shrink=10 - reduce stretchability/shrinkability (default is 20/20)
%-----------------------------------------------------------------------------------
%/\/\/\/\/\/\/\/\/\/\/\/\/\/\/\/\/\/\/\/\/\/\/\/\/\/\/\/\/\/\/\/\/\/\/\/\
%					
%------------------------------------------------------------------------
%________________________________________________________________________
%
%
%
%
%
%
%________________________________________________________________________
%------------------------------------------------------------------------
%					additional 'Specials'
%/\/\/\/\/\/\/\/\/\/\/\/\/\/\/\/\/\/\/\/\/\/\/\/\/\/\/\/\/\/\/\/\/\/\/\/\
% See PassOptions for url on Top
\usepackage[%
	pdfencoding=auto,%
	colorlinks=true,%
	urlcolor=black,%
	linkcolor=black,%
% 	citecolor=citegreen,%
	citecolor=black,%
	filecolor=magenta,%
	breaklinks,%
	]{hyperref}%
\usepackage{url}%
\usepackage{nameref}%
\usepackage{cleveref}%
%/\/\/\/\/\/\/\/\/\/\/\/\/\/\/\/\/\/\/\/\/\/\/\/\/\/\/\/\/\/\/\/\/\/\/\/\
%						END 'additional Specials'
%------------------------------------------------------------------------
%________________________________________________________________________
%
%
%
%
%
%
%________________________________________________________________________
%------------------------------------------------------------------------
%					Für Abkürzungen, Glossar und weiteres
%/\/\/\/\/\/\/\/\/\/\/\/\/\/\/\/\/\/\/\/\/\/\/\/\/\/\/\/\/\/\/\/\/\/\/\/\
% Glossaries muss NACH hyperref, babel, polyglossia, inputenc and fontenc geladen werden
% See PassOptions for fp on Top
% \usepackage[%
% 	acronym,%
% % 	nomain,%
% % 	nonumberlist,%
% 	nopostdot,%
% 	seeautonumberlist,%
% 	shortcuts,%
% 	section=chapter,%
% 	toc,%
% ]{glossaries}%
% \makeglossaries%
% \loadglsentries{./supply/glossaries.tex}%
% \glsdisablehyper% Disables Hyperlinks
%/\/\/\/\/\/\/\/\/\/\/\/\/\/\/\/\/\/\/\/\/\/\/\/\/\/\/\/\/\/\/\/\/\/\/\/\
%					glossaries fertig
%------------------------------------------------------------------------
%________________________________________________________________________
%
%
%
%
%
%
%________________________________________________________________________
%------------------------------------------------------------------------
%					Additional Packages
%/\/\/\/\/\/\/\/\/\/\/\/\/\/\/\/\/\/\/\/\/\/\/\/\/\/\/\/\/\/\/\/\/\/\/\/\

\usepackage{soul}% Viele Texthervorhebungs-Kommandos
\usepackage{lipsum}%
\usepackage{boxedminipage}%
\usepackage{ocgx2}% Package ocgx2 serves as a drop-in replacement for ocg-p, ocg and ocgxpackages.
%\usepackage{scrextend}% The pack­age makes some fea­tures of the KOMA-Script classes avail­able for other classes, e.g., for the stan­dard classes.

%/\/\/\/\/\/\/\/\/\/\/\/\/\/\/\/\/\/\/\/\/\/\/\/\/\/\/\/\/\/\/\/\/\/\/\/\
%					
%------------------------------------------------------------------------
%________________________________________________________________________
%
%
%
%
%
%
%________________________________________________________________________
%------------------------------------------------------------------------
%                   Für Zitate / Quellen:
%                   --- Begin biblatex ---
%/\/\/\/\/\/\/\/\/\/\/\/\/\/\/\/\/\/\/\/\/\/\/\/\/\/\/\/\/\/\/\/\/\/\/\/\
% 	\usepackage[%
% 		backend=biber,%
% 		style=numeric-comp,%numeric-comp, ieee
% 		isbn=false,%
% 	% 	style=apa,%
% 	% 	apabackref=true,%
% 		hyperref=true,%
% 		maxbibnames=99,%
% 		sorting=none,%
% 		natbib=true,%
% 		language=ngerman,%
% 		defernumbers=true,%
% 		]{biblatex}%
% 	% \DeclareLanguageMapping{ngerman}{ngerman-apa}%
% 	\DeclareLanguageMapping{english}{english-apa}%
% 	\addbibresource{./supply/literature.bib}%
% 	% \addbibresource{./supply/literature_secondary.bib}%
% 	% \addbibresource{./supply/literature_figures.bib}%
% 	% \addbibresource{./supply/literature_tables.bib}%
% 	% \addbibresource{./supply/literature_querries.bib}%
% 	\defbibheading{literature}{\chapnt{Literatur}{chap:literatur}\addtocounter{chapter}{-1}}%
% 	\defbibheading{literature_secondary}{\chapnt{Zusätzliche Literatur}{chap:literatur_secondary}\addtocounter{chapter}{-1}}%
% 	\defbibheading{literature_figures}{\chapnt{Abbildungs-Quellen}{chap:abb-quellen}\addtocounter{chapter}{-1}}%
% 	\defbibheading{literature_tables}{\chapnt{Tabellen-Quellen}{chap:tab-quellen}\addtocounter{chapter}{-1}}%
%========================================================================================================
%%========  Some Settings, especially for use with 'style=ieee'
% \renewcommand*{\bibfont}{\normalfont\footnotesize}
% \DeclareFieldFormat{sentencecase}{\csname bbx@colon@search\endcsname#1}% Therewith capital letters stay capitals, when using 'style=ieee'
%%======  One might also want to take a look at 'titlecase', when using 'style=ieee'
%========================================================================================================
%      Alternative Methods
%--------------------------------------------------------------------------------------------------------
% % % % % % % \DeclareBibliographyCategory{literatur}%
% % % % % % % \DeclareBibliographyCategory{literatur_abbildung}%
% % % % % % % \addtocategory{literatur}{Key1,Key2,..}%
% % % % % % % \addtocategory{literatur_abbildung}{Key1,Key4,..}%
% % % % % % \bibliography{./literatur/literatur}%
%========================================================================================================
% 	\DeclareBibliographyDriver{report}{%
  \usebibmacro{bibindex}%
  \usebibmacro{begentry}%
  \usebibmacro{author}%
  \setunit{\labelnamepunct}\newblock
  \usebibmacro{title}%
  \newunit
  \printlist{language}%
  \newunit\newblock
  \usebibmacro{byauthor}%
  \newunit\newblock
  \printfield{type}%
  \setunit*{\addspace}%
  \printfield{number}%
  \newunit\newblock
  \printfield{version}%
  \newunit
  \printfield{note}%
  \newunit\newblock
  \usebibmacro{institution+location+date}%
  \newunit\newblock
  \usebibmacro{chapter+pages}%
  \newunit
  \printfield{pagetotal}%
  \newunit\newblock
  \iftoggle{bbx:isbn}
    {\printfield{isrn}}
    {}%
  \newunit\newblock
  \usebibmacro{doi+eprint+url}%
  \newunit\newblock
  \usebibmacro{addendum+pubstate}%
  \setunit{\bibpagerefpunct}\newblock
  \usebibmacro{pageref}%
  \newunit\newblock
  \iftoggle{bbx:related}
    {\usebibmacro{related:init}%
     \usebibmacro{related}}
    {}%
  \usebibmacro{finentry}}
%
\renewbibmacro*{institution+location+date}{%
  \printlist{location}%
  \iflistundef{institution}
    {\setunit*{\addcomma\space}}
    {\setunit*{\addcolon\space}}%
  \textsf{\printlist{institution}}%
  \setunit*{\adddot\space}%
  \usebibmacro{date}%
  \newunit}
%
\renewbibmacro*{publisher+location+date}{%
  \printlist{location}%
  \iflistundef{publisher}
    {\setunit*{\addcomma\space}}
    {\setunit*{\addcolon\space}}%
  \printlist{publisher}%
  \setunit*{\adddot\space}%
  \usebibmacro{date}%
  \newunit}
%
\DeclareBibliographyDriver{misc}{%
  \usebibmacro{bibindex}%
  \usebibmacro{begentry}%
  \usebibmacro{author/editor+others/translator+others}%
  \setunit{\labelnamepunct}\newblock
  \usebibmacro{title}%
  \newunit
  \printlist{language}%
  \newunit\newblock
  \usebibmacro{byauthor}%
  \newunit\newblock
  \usebibmacro{byeditor+others}%
  \newunit\newblock
  \printfield{howpublished}%
  \newunit\newblock
  \printfield{type}%
  \newunit
  \printfield{version}%
  \newunit
  \printfield{note}%
  \newunit\newblock
  \usebibmacro{organization+location+date}%
  \newunit\newblock
  \usebibmacro{doi+eprint+url}%
  \newunit\newblock
  \usebibmacro{addendum+pubstate}%
  \setunit{\bibpagerefpunct}\newblock
  \usebibmacro{pageref}%
  \newunit\newblock
  \iftoggle{bbx:related}
    {\usebibmacro{related:init}%
     \usebibmacro{related}}
    {}%
  \usebibmacro{finentry}}
%
\renewbibmacro*{organization+location+date}{%
  \printlist{location}%
  \iflistundef{organization}
    {\setunit*{\addcomma\space}}
    {\setunit*{\addcolon\space}}%
  \textsf{\printlist{organization}}%
  \setunit*{\addcomma\space}%
  \usebibmacro{date}%
  \newunit}
%
%
%==========================================================================================
%------------------------------------------------------------------------------------------
%------------------------------------------------------------------------------------------
%__________________________________________________________________________________________
%
\DefineBibliographyStrings{ngerman}{andothers={et\addabbrvspace al\adddot}}
%
% \renewcommand*{\bibpagespunct}{\addcolon}      % Zeichen vor der Seitenangabe
\renewcommand*{\multinamedelim}{\addcomma\space}   % Kommas zwischen Autoren [default]
\renewcommand*{\finalnamedelim}{\addcomma\space}   % Entfernt das 'und' zwischen den letzten beiden Namen einer Liste
\renewcommand*{\nameyeardelim}{\addcomma\space}	% Bei Zitaten Komma vor Jahr
\defbibheading{quell}{\section{Quellenverzeichnis}}
\defbibheading{lit}{\section{Literaturverzeichnis}}
\setcounter{biburllcpenalty}{9000}
\setcounter{biburlucpenalty}{9000}%Allow URLs to break, if run into margin%
%/\/\/\/\/\/\/\/\/\/\/\/\/\/\/\/\/\/\/\/\/\/\/\/\/\/\/\/\/\/\/\/\/\/\/\/\
%                   --- End biblatex ---
%------------------------------------------------------------------------
%________________________________________________________________________
%
%
%
%
%
%
%
%
%
%
%

% Makro to define a chapter without numbering, but with Entry in Table of Contents
% Chapter without numbering to TOC
% Arguments:
% #1 - Optional. Alternative Name for TOC
% #2 - Mandatory. Name of the Chap/Sec/Subsec itself
% #3 - Optional. Label
%
%					%				Old Version					%
% 					% 					% \newcommand{\chapnt}[2]{%
% 					% 					% 	\chapter*{#1}%
% 					% 					% 	\label{#2}%
% 					% 					% 	\addcontentsline{toc}{chapter}{\nameref{#2}}%
% 					% 					% 	\markboth{\nameref{#2}}{\nameref{#2}}%
% 					% 					% 	\addtocounter{chapter}{1}%
% 					% 					% }%
\DeclareDocumentCommand{\chapnt}{%
O{#2} m o%
}{%
	\chapter*{#2}%
	\IfValueTF{#3}{%
		\label{#3}%
	}{}%
% 	\addcontentsline{toc}{chapter}{\nameref{#3}}%
% 	\markboth{\nameref{#2}}{\nameref{#3}}%
	\addcontentsline{toc}{chapter}{#1}%
	\markboth{#1}{#1}%
	\addtocounter{chapter}{1}%
}%
% Usage: \chapnt{Chapter Name}{Label}
%
% And for Section
%	Sets the \rightmark. Remember to change the Brackets for \leftmark if needed
%	With empty Bracket it sets it blank
%					%				Old Version					%
% 					% 					% \newcommand{\secnt}[2]{%
% 					% 					% 	\section*{#1}%
% 					% 					% 	\label{#2}%
% 					% 					% 	\addcontentsline{toc}{section}{\nameref{#2}}%
% 					% 					% 	\markboth{}{\nameref{#2}}%
% 					% 					% }%
\DeclareDocumentCommand{\secnt}{%
O{#2} m o%
}{%
	\section*{#2}%
	\IfValueTF{#3}{%
		\label{#3}%
	}{}%
	\addcontentsline{toc}{section}{#1}%
% 	\markboth{\leftmark}{#1}%
	\markright{#1}%
}%
%
% And for SubSection
%					%				Old Version					%
% 					% 					% \newcommand{\subsecnt}[2]{%
% 					% 					% 	\subsection*{#1}%
% 					% 					% 	\label{#2}%
% 					% 					% 	%\addcontentsline{toc}{subsection}{\protect\numberline{}#1}%This would add a whitespace, where normally the section-numbering would be.
% 					% 					% 	\addcontentsline{toc}{subsection}{#1}%
% 					% 					% 	\markboth{}{\nameref{#2}}%
% 					% 					% }%
\DeclareDocumentCommand{\subsecnt}{%
O{#2} m o%
}{%
	\subsection*{#1}%
	\IfValueTF{#3}{%
		\label{#3}%
	}{}%
	%\addcontentsline{toc}{subsection}{\protect\numberline{}#1}%This would add a whitespace, where normally the section-numbering would be.
	\addcontentsline{toc}{subsection}{#1}%
}%
%
% Chapter with Subtitle
\newcommand\chaptersubt[2]{\chapter%
  [#1\hfil\hbox{}\protect\linebreak{\itshape#2}]%
  {#1\\[2ex]\Large\itshape#2}%
}
\newcommand\sectionsubt[2]{\section%
  [#1\hfil\hbox{}\protect\linebreak{\itshape#2}]%
  {#1\\[2ex]\Large\itshape#2}%
}
%
%
%
%
%
%
% own command to get a newline, without this disturbing underfull \hbox
% still beware of the "Glue" from the set \parskip height plus x minus y
% which in fact allows latex to stretch the space between paragraphs
% (by default up to 1pt)
\newcommand{\nl}{\par\noindent} % Abk. für NewLine
% and here a command to create really a new paragraph
% inserts an additional free line
\newcommand{\np}{\par\vspace{\baselineskip}} % Abk. für NewParagraph
\newcommand{\npi}{\par\vspace{\baselineskip}\noindent} % Abk. für NewParagraph_noIndent
% Zu benutzen also so:
% Neue Zeile ohne Einrückung: \nl
% Neue Zeile mit Einrückung: Freie Zeile im Editor
% Neuer Absatz mit freier Zeile dazwischen: \np
\newcommand{\greenuline}[1]{\colorlet{temp}{.}\color{green}\underline{\color{temp}#1}\color{temp}\xspace}%
%
%
%
%
%WorkingStateDivider
\newcommand{\wdiv}{\vspace{0.5\baselineskip}\npi\leavevmode\xleaders\hbox{=}\hfill\kern0pt\nl\#\#\#\#\#\#\#\#\#\#\#\#\#\#\#\#\#\#\#\#\#\#\#\#\leavevmode\xleaders\hbox{ ・ }\hfill\kern0pt\#\#\#\#\#\#\#\#\#\#\#\#\#\#\#\#\#\#\#\#\#\#\#\#\nl}
%
%
%
% Eventuell Labelprüfung vor \cref@gettype einbauen:
%	(ifcsdef aus der etoolbox)
% Makro zur einheitlichen Formatierung von Abbildungs-Referenzen
\newcommand{\abbref}[1]{Abbildung~\ref{#1}}%
\newcommand{\abbrefzwei}[2]{Abbildungen~\ref{#1} \& \ref{#2}}%
% Und das gleiche inklusive \nameref
% Makro zur einheitlichen Formatierung von Abschnitts-Referenzen
\newcommand{\abnamref}[1]{Abbildung~\ref{#1} (\nameref{#1})}%
\newcommand{\absref}[1]{\cref{#1}}%
% \newcommand{\absref}[1]{\textbf{Abschnitt~\ref{#1}}}%
% Und das gleiche inklusive \nameref
\makeatletter%
\newcommand{\absamref}[1]{%
 	\cref@gettype{#1}{\currlabtype}%
	\IfSubStr{\currlabtype}{part}{%
		\textsf{\cref{#1} - \nameref{#1}}%
	}{%
	\IfSubStr{#1}{subsubsec}{%
%	'subsubsection' isn't functional with gettype.
%	This only goes deep until subsection
%	So be careful and append a preceding 'subsubsec'
%	on the label
		Unterabschnitt >>\nameref{#1}<< aus \ref{#1}%
	}{%
		\cref{#1} (\nameref{#1})%
	}%
	}%
}%
\makeatother%
% \newcommand{\absamref}[1]{\textbf{Abschnitt~\ref{#1}} (\nameref{#1})}%
% Für 2 Abschnitte zugleich
\makeatletter%
\newcommand{\absrefzwei}[2]{%
	\cref@gettype{#1}{\currlabtype}%
	\IfSubStr{\currlabtype}{chapter}{%
		Kapiteln~\ref{#1} \& \ref{#2}%
	}{%
		Abschnitten~\ref{#1} \& \ref{#2}%
	}%
}%
\makeatother%
\makeatletter%
\newcommand{\absamrefzwei}[2]{%
	\cref@gettype{#1}{\currlabtype}%
	\IfSubStr{\currlabtype}{chapter}{%
		Kapiteln~\ref{#1} (\nameref{#1}) \&
			\ref{#2} (\nameref{#2})%
	}{%
		Abschnitten~\ref{#1} (\nameref{#1}) \&
			\ref{#2} (\nameref{#2})%
	}%
}%
\makeatother%
% Absatz Name Reference Zwei Short -- means without the subsequent 'n' -
% i.e. Kapitel instead of Kapiteln
\makeatletter%
\newcommand{\absamrefzweis}[2]{%
	\cref@gettype{#1}{\currlabtype}%
	\IfSubStr{\currlabtype}{chapter}{%
		Kapitel~\ref{#1} (\nameref{#1}) \&
			\ref{#2} (\nameref{#2})%
	}{%
		Abschnitte~\ref{#1} (\nameref{#1}) \&
			\ref{#2} (\nameref{#2})%
	}%
}%
\makeatother%
% \newcommand{\absrefzwei}[2]{\textbf{Abschnitten~\ref{#1}} \& \ref{\ref{#2}}}%
% \newcommand{\absamrefzwei}[2]{\textbf{Abschnitten~\ref{#1}} (\nameref{#1}) \& \textbf{\ref{#2}} (\nameref{#2})}%
%
% 
%
%
%
% 
% Makro für Kreis um Text
\newcommand\circlearound[1]{%
  \tikz[baseline]\node[draw,shape=circle,anchor=base,inner sep=0.3ex] {#1} ;}
\newcommand\lipsaround[1]{%
  \tikz[baseline]\node[draw,shape=ellipse,anchor=base,inner sep=0.3ex] {#1} ;}
%
\newcommand\rectdotaround[2]{%
  \tikz[baseline]\node[draw=#1,thick, loosely dotted,%
  shape=rectangle,anchor=base,inner sep=0.5ex] {#2} ;}
%
%
%
%
%Some Symbols. (Using package 'amssymb', 'pifont')
\newcommand{\checksign}{\ding{51}}%
\newcommand{\crosssign}{\ding{55}}%
%
%
%
%
%
%
%=================================
% The romaji Spoiler-Environment
%      requires package 'ocgx' / 'ocgx2'
%================================================
%	Button at Start of Env
\newcounter{romaji}%
\newenvironment{romaji}%
{\stepcounter{romaji}%
\begin{minipage}{\textwidth-1ex}%
\switchocg{romaji\arabic{romaji}}{\fbox{\tiny ローマ字}}\begin{ocg}{Romaji \arabic{romaji}}{romaji\arabic{romaji}}{0}%
\footnotesize}%
%---------------------------------------------
{\par\end{ocg}%
\end{minipage}}%
%================================================
%	Button at End of Env
% 		\newcounter{romaji}%
% 		\newenvironment{romaji}%
% 		{\stepcounter{romaji}%
% 		\begin{minipage}{\textwidth-1ex}%
% 		\begin{ocg}{Romaji \arabic{romaji}}{romaji\arabic{romaji}}{0}%
% 		\footnotesize}%
% 		%---------------------------------------------
% 		{\end{ocg}\switchocg{romaji\arabic{romaji}}{\fbox{\tiny ローマ字}}%
% 		\end{minipage}}%
%================================================
\newcommand{\ropre}{\par\vspace{0.2\baselineskip}\noindent}% Romaji-Pre
%
%
%
%
%
%
%
%
%
%
%
%
%
%
%________________________________________________________________________
%------------------------------------------------------------------------
%			Japanische Schrift
%/\/\/\/\/\/\/\/\/\/\/\/\/\/\/\/\/\/\/\/\/\/\/\/\/\/\/\/\/\/\/\/\/\/\/\/\
% \makeatletter%
% \newcommand{\jap}[1]{\begin{CJK}{UTF8}{min}#1\end{CJK}}%
% \makeatother%
%/\/\/\/\/\/\/\/\/\/\/\/\/\/\/\/\/\/\/\/\/\/\/\/\/\/\/\/\/\/\/\/\/\/\/\/\
%							Japanisch fertig
%------------------------------------------------------------------------
%________________________________________________________________________
%
%
%
%________________________________________________________________________
%------------------------------------------------------------------------
%			Verschiedene Definitionen wie Schriftzeichen
%/\/\/\/\/\/\/\/\/\/\/\/\/\/\/\/\/\/\/\/\/\/\/\/\/\/\/\/\/\/\/\/\/\/\/\/\
\newcommand{\mytilde}{{\raise.17ex\hbox{$\scriptstyle\mathtt{\sim}$}}}
\newcommand{\mytildeA}{\textasciitilde}
\newcommand{\mytildeB}{$\sim$}
\newcommand{\mytildeC}{\~{}}
\newcommand{\mytildeD}{\texttildelow}
\newcommand{\mytildeE}{\raisebox{-.8ex}{\textasciitilde}}
%/\/\/\/\/\/\/\/\/\/\/\/\/\/\/\/\/\/\/\/\/\/\/\/\/\/\/\/\/\/\/\/\/\/\/\/\
%							Definitionen fertig
%------------------------------------------------------------------------
%________________________________________________________________________
%
%
%
%
%
%
%
%
%
% Optionen für fontseries und fontshape:
% Series, any combination of weight and width is [in theory] possible:
% weight                    width
% Ultra Light       ul      Ultra Condensed     uc    
% Extra Light       el      Extra Condensed     ec    
% Light             l       Condensed            c      
% Semi Light        sl      Semi Condensed      sc    
% Medium (normal)   m
% Semi Bold         sb      Semi Expanded       sx    
% Bold              b       Expanded             x 
% Extra Bold        eb      Extra Expanded      ex 
% Ultra Bold        ub      Ultra Expanded      ux
% %
% Shape:
% upright (normal)   n 
% italic             it
% slanted/oblique    sl 
% small caps         sc
% upright italic     ui
% outline            ol 
%
%Inline English
%Zum anders formatieren von nicht übersetzten, englischen Begriffen
\newcommand*{\ien}[1]{%
	#1%
% 	{\small\fontfamily{phv}\fontseries{m}\fontshape{n}\selectfont#1}%
}%
%
%Inline Code
%Zum eigens formatieren von kleinen Code-Fetzen
\newcommand*{\ico}[1]{%
	\texttt{#1}%
}%
%
%
%
%
% Kommando-Zeilen Syntax
% The one with the 'b' suffix is for brackets only. It does not format anything,
% but add the brackets. Its something like for "inner-use". You can use it in
% wrapped makros for example.
\newcommand*{\regexcmd}[2]{% Regular Expression Command Line
	\IfSubStr{#1}{id}{%
		\texttthyph{\textcolor{ColorRegExCmdID}{<#2>}}%
	}{\IfSubStr{#1}{auswahl}{%
		\texttthyph{\textcolor{ColorRegExCmdSelect}{[#2]}}%
	}{\IfSubStr{#1}{mehrfach}{%
		\texttthyph{\textcolor{ColorRegExCmdMultiSelect}{\{#2\}}}%
	}{\IfSubStr{#1}{option}{%
		\texttthyph{-\textcolor{ColorRegExCmdOption}{$\|$#2$\|$}}%
	}{%
	}}}}%
}%
\newcommand*{\regexcmdb}[2]{% Regular Expression Command Line
	\IfSubStr{#1}{id}{%
		<#2>%
	}{\IfSubStr{#1}{auswahl}{%
		[#2]%
	}{\IfSubStr{#1}{mehrfach}{%
		\{#2\}%
	}{\IfSubStr{#1}{option}{%
		-$\|$#2$\|$%
	}{%
	}}}}%
}%
%
%
%
%
%
\definecolor{marginparcolor}{RGB}{31,73,125}%
% Neuer MarginPar, für Schriftgröße
\newcommand*{\Marginpar}{}% Anweisung "reservieren"
% Das "Standard-Kommando" in neuem Kommando mit großem 'M' retten:
\let\Marginpar\marginpar% \MarginPar ist jetzt dasselbe wie \marginpar
\renewcommand*{\marginpar}[2][]{{% ein optionales und ein normales Argument
\expandafter\renewcommand*{\glstextformat}[1]{\color{marginparcolor}##1}%
  \ifstr{#1}{}{%
  	\Marginpar{\footnotesize\raggedright \textcolor{marginparcolor}{#2}}%
  }{%
    \Marginpar[{\footnotesize\raggedright \textcolor{marginparcolor}{#1}}]%
    	{\footnotesize\raggedright \textcolor{marginparcolor}{#2}}%
  }%
}}% Double Braces to create a scope for the redefinition of glstextformat
%
%
%
%
%
%________________________________________________________________________
%------------------------------------------------------------------------
%							Makros für Cleveref
%/\/\/\/\/\/\/\/\/\/\/\/\/\/\/\/\/\/\/\/\/\/\/\/\/\/\/\/\/\/\/\/\/\/\/\/\
%Get Type of \ref with help of cleveref
\def\currlabtype{}% Variable reservieren, um im ChapRef Makro den Typ zu speichern
%/\/\/\/\/\/\/\/\/\/\/\/\/\/\/\/\/\/\/\/\/\/\/\/\/\/\/\/\/\/\/\/\/\/\/\/\
%							Cleveref fertig
%------------------------------------------------------------------------
%________________________________________________________________________
%
%
%
%
%
\newcommand{\tikzpicturescale}{1.0}
\newcommand{\tikzpicturescaleboxfactor}{1.0}
\newcommand{\stdleftbarwidth}{0.2em}
% 
%
%
%
%
% INFO:
% To print a value (length, printlength)
% \the\value
% e.g.: \the\textwidth
% e.g.: \the\linewidth
% e.g.: \the\columnwidth
%
%
%
%
% tikz makros
%
% Variable für die Breite/Höhe von eingebundenen Bildern
% Verwendet im folgenden Makro
\newlength{\tikzwidth}%
\newlength{\tikzheight}%
\newlength{\textheightscaled}%
\newlength{\textwidthactual}%
\newcommand{\tikzHeightFactorArgument}{1}%
\newcommand{\tikzWidthFactorArgument}{1}%
\newlength{\tikzAimedHeight}%
\newlength{\tikzAimedWidth}%
\newlength{\captionHeight}%
\newcommand{\tikzFloatPositioning}{!ht}
% \textheightscaled=0.96\textheight%
% Makro für die Tikz Abbildungen
% Arg#1 (Optional): Includestandalone-Mode
% Arg#2: Pic-Name (Location: \tikzFilesPath/NAME.tex) % This Macro can be defined somewhere, by you. In my Template it is found inside the Preamble right by the inclusion of the tikz-Package, together with \includestandalonedefaultmode
% Arg#3 (Optional): Alternatives Caption (TableOfContents usw.)
% Arg#4: Bildbeschriftung
% Arg#5: Number of Lines des Caption
% Arg#6 (Optional): Liste mit folgendem Inhalt. Parameter zu Skalierung und Positionierung
%			#1: Faktor der Seitenhöhe, der nicht überschritten werden soll 
%			#2: Faktor der Seitenbreite, der nicht überschritten werden soll
%			#3: Positionierungsmodus. Bsp.: !ht, !htpb, H
%	  Beispiel für ein Listen-Format: [1,1,!htp]
% Arg#7: Label für referencing
% Arg#8 (Optional): Rotations-Winkel für das Bild
%% Möglichkeiten für den includestandalone-mode: tex, buildnew
%% The Default Macro: See Definition in Header (should be buildnew)
%% Use the optinal Argument with "tex", to make the tikz-Pics inline and hence
%% compiled every Runtime, without pre-rendered picture. buildnew creates
%% pre-rendered Pictures and includes them like standard-png; only rendered, if
%% tex-file is newer than the picture
\DeclareDocumentCommand{\tikzabb}{%
O{\includestandalonedefaultmode} m o m m >{\SplitList{,}}O{1,1,!ht} m O{0}%
}{%
\def\efigure{\begin{figure}}%
\def\efigureend{\end{figure}}%
\SplitListScalePos#6\relax%
%\textheightscaled=\textheight-\parskip-\abovecaptionskip-\belowcaptionskip-\baselineskip%
\textheightscaled=\dimexpr\textheight-#5\baselineskip-\parskip-\abovecaptionskip-\belowcaptionskip\relax%
\textwidthactual=\dimexpr\linewidth\relax%
\ifthenelse{#8=90}{%
	\tikzAimedHeight=\tikzWidthFactorArgument\textwidthactual%
	\tikzAimedWidth=\tikzHeightFactorArgument\textheightscaled%
}{
	\ifthenelse{#8=-90}{%
		\tikzAimedHeight=\tikzWidthFactorArgument\textwidthactual%
		\tikzAimedWidth=\tikzHeightFactorArgument\textheightscaled%
	}{
		\tikzAimedHeight=\tikzHeightFactorArgument\textheightscaled%
		\tikzAimedWidth=\tikzWidthFactorArgument\textwidthactual%
	}%
}%
\settowidth{\tikzwidth}{\includestandalone{\tikzFilesPath#2}}%
\settoheight{\tikzheight}{\includestandalone{\tikzFilesPath#2}}%
\tikzwidth=\tikzpicturescaleboxfactor\tikzwidth%
\tikzheight=\tikzpicturescaleboxfactor\tikzheight%
\ifthenelse{\tikzAimedHeight<\tikzheight}{%
	\tikzheight=\tikzAimedHeight%
	\settowidth{\tikzwidth}{\includestandalone[height=\tikzheight]{\tikzFilesPath#2}}%
	\ifthenelse{\tikzAimedWidth<\tikzwidth}{%
		\tikzwidth=\tikzAimedWidth%
		\expandafter\efigure\expandafter[\tikzFloatPositioning]%
			\centering%
% 	 		\fbox{%
			\includestandalone[width=\tikzwidth,angle=#8,mode=#1]{\tikzFilesPath#2}%
% 	 		}%
			\IfValueTF{#3}{%
				\caption[#3]{#4}%
			}{%
				\caption{#4}%
			}%
			\label{#7}%
		\efigureend%
	}{%
		\expandafter\efigure\expandafter[\tikzFloatPositioning]%
			\centering%
% 	 		\fbox{%
			\includestandalone[height=\tikzheight,angle=#8,mode=#1]{\tikzFilesPath#2}%
% 	 		}%
			\IfValueTF{#3}{%
				\caption[#3]{#4}%
			}{%
				\caption{#4}%
			}%
			\label{#7}%
		\efigureend%
	}%
}{%
	\ifthenelse{\tikzAimedWidth<\tikzwidth}{\tikzwidth=\tikzAimedWidth}{}%
	\expandafter\efigure\expandafter[\tikzFloatPositioning]%
		\centering%
% 		\fbox{%
		\includestandalone[width=\tikzwidth,mode=#1]{\tikzFilesPath#2}%
% 		}%
		\IfValueTF{#3}{%
			\caption[#3]{#4}%
		}{%
			\caption{#4}%
		}%
		\label{#7}%
	\efigureend%
}%
}%
%
\newcommand{\SplitListScalePos}[3]{%
	\renewcommand{\tikzHeightFactorArgument}{#1}%
	\renewcommand{\tikzWidthFactorArgument}{#2}%
	\renewcommand{\tikzFloatPositioning}{#3}%
}%
%
% Nice Makro, which allows to check if a node with certain name exists.
% Gives you an If-Then-Else Operator for use like this:
% \ifnodedefined{Node-name without brackets}{Wenn existent, dann das}{else}
\makeatletter%
\long\def\ifnodedefined#1#2#3{%
  \@ifundefined{pgf@sh@ns@#1}{#3}{#2}}%
%
\newcommand\aeundefinenode[1]{%%
  \expandafter\ifx\csname pgf@sh@ns@#1\endcsname\relax%
  \else%
    \typeout{===>Undefining node "#1"}%
    \global\expandafter\let\csname pgf@sh@ns@#1\endcsname\relax%
  \fi%
}%
%Another to undefine nodes, that they are gone in successive pics
\newcommand\aeundefinethesenodes[1]{%
  \foreach \myn  in {#1}%
    {%
      \expandafter\aeundefinenode\expandafter{\myn}%
    }%
}%
\makeatother%
%
%
% Own redefinition of leftbar environment
% Einfach ums schöner zu machen
\newenvironment{myleftbar}[2]%[1=0.5pt,2=5pt]%
{\setlength{\topsep}{0ex}%
\def\FrameCommand{\hspace{#2} \vrule width #1 \hspace{0.0em}}%
\MakeFramed%
{\advance\hsize-\width \FrameRestore}}%
{\endMakeFramed}%
%
%
\newenvironment{scope}{}{}
%
%
%
%
% Own Makro für footnote ohne Marker
\newcommand\footnoteblank[1]{%
  \begingroup
  \renewcommand\thefootnote{}\footnote{#1}%
  \addtocounter{footnote}{-1}%
  \endgroup
}
%
%
%
%
%
%
%==========================================
%##########################################
%    Schriftgrößen
%==========================================
\tiny
\scriptsize
\footnotesize
\small
\normalsize
\large
\Large
\LARGE
\huge
\Huge
%##########################################
%==========================================%
%
% Extrusion Makros (needs shape=rectangleRoundedAnchorSurround)
% You have to specify a coordinate-Node as Base-Anchor and pass it
% Then pass the Directions
% Styles have to contain at least:
%	minimum width, minimum height, rounded corners attributes=RoundedCornersAmount
% Arguments:
% {AnchorNode}
% {up / down} %Not used for now
% {left / right} %Not used for now
% {StyleFront}
% {StyleBack}
% {FrontColor}
% {BackColor}
% {RoundedCornersAmount}
% Later on added:
% {ShiftAmountX}
% {SjoftAmountY}
%
\DeclareDocumentCommand{\ExtrudeOutRoundedTop}{ m m m m m m m m }{%
	\node(#1back)[#5,fill=#7,anchor=south]at(#1){};%
	\node(#1front)[#4,fill=#6,xshift=-2mm,yshift=2mm,anchor=south]at(#1){};%
	\path[draw=black,fill=#7](#1front.south west)--(#1front.south east)--%
		(#1back.south east)-- (#1back.south west)--(#1front.south west);%
	\path[shading=axis,%
		top color=#6,bottom color=#7,%
% 		middle color=#6,%
		shading angle=45,%
		transform canvas={shift={(-0.5\pgflinewidth,0)}},%
		]%
		($(#1front.east|-#1front.north)+(-#8,0)$)to[out=0,in=90]%
		($(#1front.east|-#1front.north)+(0,-#8)$)to%
		($(#1back.east|-#1back.north)+(0,-#8)$)to[out=90,in=0]%
		($(#1back.east|-#1back.north)+(-#8,0)$)to%
		($(#1front.east|-#1front.north)+(-#8,0)$);%
% 	\path[draw=black]%
% 		($(#1front.east|-#1front.north)+(-#8,0)$)to[out=0,in=90]%
% 		($(#1front.east|-#1front.north)+(0,-#8)$);%
% 	\path[draw=black]%
% 		($(#1back.east|-#1back.north)+(0,-16pt)$)to%
% 		($(#1back.east|-#1back.north)+(0,-#8)$)to[out=90,in=-45]%
%  		(#1back.north east);%
% 		($(#1back.east|-#1back.north)+(-#8,0)$)to%
% 		($(#1back.east|-#1back.north)+(-16pt,0)$);%
	\node(#1front2)[#4,fill=#6,anchor=south]at(#1front.south){};%
}%
%
\DeclareDocumentCommand{\ExtrudeOutRoundedBottom}{ m m m m m m m m }{%
	\node(#1back)[#5,fill=#7,anchor=north]at(#1){};%
	\node(#1front)[#4,fill=#6,xshift=-2mm,yshift=2mm,anchor=north]at(#1){};%
	\path[fill=abifacebackcolor](#1front.north east)--%
		(#1back.north east)--%
		(#1back.north east-|#1front.north east)--%
		(#1front.north east);%
		\draw[draw=abifacebackcolor,line width=0.8pt]%
			($(#1back.north east)+(0,-0.25\pgflinewidth)$)--%
			($(#1back.north east-|#1front.north east)+(0,-0.25\pgflinewidth)$);%
		\draw(#1front.north east)--(#1back.north east);%
		\draw[](#1back.north east)--%
			($(#1back.north east)+(0,-0.6pt)$);%
	\path[shading=axis,%
		left color=abifacebackcolor,right color=abifacebackcolor,%
		middle color=abifacecolor,%
		shading angle=135,]%
		($(#1front.east|-#1front.south)+(0pt,#8)$)to[out=-90,in=0]%
		($(#1front.east|-#1front.south)+(-#8,0pt)$)to%
		($(#1back.east|-#1back.south)+(-#8,0pt)$)to[out=0,in=-90]%
		($(#1back.east|-#1back.south)+(0pt,#8)$)to%
		($(#1front.east|-#1front.south)+(0pt,#8)$);%
	\path[draw=black]%
		($(#1back.east|-#1back.south)+(-16pt,0pt)$)to%
		($(#1back.east|-#1back.south)+(-#8,0pt)$)to[out=0,in=-90]%
		($(#1back.east|-#1back.south)+(0pt,#8)$)to%
		($(#1back.east|-#1back.south)+(0pt,16pt)$);%
	\path[shading=axis,%
		bottom color=#6,top color=#7,%
% 		middle color=#6,%
		shading angle=-135,%
		transform canvas={shift={(0,0.5\pgflinewidth)}},%
		]%
		($(#1front.west|-#1front.south)+(0,#8)$)to[out=-90,in=0]%
		($(#1front.west|-#1front.south)+(#8,0)$)to%
		($(#1back.west|-#1back.south)+(#8,0)$)to[out=180,in=-90]%
		($(#1back.west|-#1back.south)+(0,#8)$)to%
		($(#1front.west|-#1front.south)+(0,#8)$);%
	\node(#1front2)[#4,fill=#6,anchor=north]at(#1front.north){};%
}%%
%
%
%
% Setups
%
\interfootnotelinepenalty=6000 %Gibt die Dringlichkeit an, mit der Fussnoten
					%nicht umgebrochen werden (bis 10000)
					%Standard: 100
%
% \newcolumntype{L}{>{\raggedright\arraybackslash}X}
%\setlength{\parindent}{0em}
%\setlength{\parskip}{0ex plus 0pt minus 0pt}
% Default value is 0pt plus 1pt
%Latex Std is \setlength\parskip{0\p@ \@plus \p@}
%
%
%\tikzset{
%	fontscale/.style = {font=\relsize{#1}}
%	}
%\counterwithin{figure}{section}
% PSTricks standart umgebung:
%\psset{xunit=0.5\textwidth,yunit=0.5\textwidth,runit=0.5\textwidth}
%
%
%
%
%
%________________________________________________________________________
%------------------------------------------------------------------------
%		Spacing für
%					- Equation Environment
%					- floats
%					- Some new Lengths & Variables
%/\/\/\/\/\/\/\/\/\/\/\/\/\/\/\/\/\/\/\/\/\/\/\/\/\/\/\/\/\/\/\/\/\/\/\/\
%	Is found in
%		%
%================================================================
%----------------------------------------------------------------
%      Setup, Typographisch
%             (Für Präambel. Titel, Verzeichnisse...)
%================================================================
%
\pagestyle{empty}%
% Hm, settings vlt. doch besser in Präambel
% %
%
% Setups
%
\interfootnotelinepenalty=6000 %Gibt die Dringlichkeit an, mit der Fussnoten
					%nicht umgebrochen werden (bis 10000)
					%Standard: 100
%
% \newcolumntype{L}{>{\raggedright\arraybackslash}X}
%\setlength{\parindent}{0em}
%\setlength{\parskip}{0ex plus 0pt minus 0pt}
% Default value is 0pt plus 1pt
%Latex Std is \setlength\parskip{0\p@ \@plus \p@}
%
%
%\tikzset{
%	fontscale/.style = {font=\relsize{#1}}
%	}
%\counterwithin{figure}{section}
% PSTricks standart umgebung:
%\psset{xunit=0.5\textwidth,yunit=0.5\textwidth,runit=0.5\textwidth}
%
%
%
%
%
%________________________________________________________________________
%------------------------------------------------------------------------
%		Spacing für
%					- Equation Environment
%					- floats
%					- Some new Lengths & Variables
%/\/\/\/\/\/\/\/\/\/\/\/\/\/\/\/\/\/\/\/\/\/\/\/\/\/\/\/\/\/\/\/\/\/\/\/\
%	Is found in
%		\input{./organization/TitlePage_&_Init.tex}
% ( '\setlength' must come after \begin{document} )
%/\/\/\/\/\/\/\/\/\/\/\/\/\/\/\/\/\/\/\/\/\/\/\/\/\/\/\/\/\/\/\/\/\/\/\/\
%			Spacing End
%------------------------------------------------------------------------
%________________________________________________________________________
%
%
%
%
%
%
%
%
%
%________________________________________________________________________
%------------------------------------------------------------------------
%							Setups für Microtype
%/\/\/\/\/\/\/\/\/\/\/\/\/\/\/\/\/\/\/\/\/\/\/\/\/\/\/\/\/\/\/\/\/\/\/\/\
% \SetProtrusion{encoding={*},family={bch},series={*},size={6,7}}
%               {1={ ,750},2={ ,500},3={ ,500},4={ ,500},5={ ,500},
%                6={ ,500},7={ ,600},8={ ,500},9={ ,500},0={ ,500}}
%%%%%%%%%%%%%%%%%%%%%%%%%%%%%%%%%%%%%%%%%%%%%%%%%%%%%%%%%%%%%%%%%
% Tracking, Spacing & Kerning werden seit Lua(La)Tex von 'fontspec' verwaltet
% 				% \SetExtraKerning[unit=space]
% 				%     {encoding={*}, family={bch}, series={*}, size={footnotesize,small,normalsize}}
% 				%     {\textendash={400,400}, % en-dash, add more space around it
% 				%      "28={ ,150}, % left bracket, add space from right
% 				%      "29={150, }, % right bracket, add space from left
% 				%      \textquotedblleft={ ,150}, % left quotation mark, space from right
% 				%      \textquotedblright={150, }} % right quotation mark, space from left
% 				% \SetTracking{encoding={*}, shape=sc}{40}
%\/\/\/\/\/\/\/\/\/\/\/\/\/\/\/\/\/\/\/\/\/\/\/\/\/\/\/\/\/\/\/\/\/\/\/\/
%							Microtype fertig
%------------------------------------------------------------------------
%________________________________________________________________________
%
%
%
%
%
%
%
%________________________________________________________________________
%------------------------------------------------------------------------
%							Setups für Glossaries
%/\/\/\/\/\/\/\/\/\/\/\/\/\/\/\/\/\/\/\/\/\/\/\/\/\/\/\/\/\/\/\/\/\/\/\/\
%Formatierung für Abkürzungen bei erstem Auftreten und weiteren und
%Falls deutsch Key im GLossar vorhanden, benutze ihn beim ersten Mal
% \defglsentryfmt[\acronymtype]{
%   \ifglsused{\glslabel}
%     {\glsgenentryfmt}% Wenn verwendet normales format
%     {\textit{\glsgenentryfmt}}% beim ersten mal kursiv
% }
% 		\newcommand*{\glsprintmitdeutsch}{%
% 			{\glsentrylong{\glslabel} (\glsentryname{\glslabel}, dt.:
% 			\glsentrydeutsch{gls-\glslabel})}%
% 		}%
% 		\newcommand*{\glsprintmitenglisch}{%
% 			{\glsentrylong{\glslabel} (\glsentryname{\glslabel}, engl.:
% 			\glsentryenglisch{gls-\glslabel})}%
% 		}%
% 		\newcommand*{\glsprintmitdualentry}{%
% 		  \ifglsused{\glslabel}%
% 		    {%
% 		    	\glsgenentryfmt%
% 		    }{%
% 		    	\ifglshasfield{deutsch}{gls-\glslabel}{%
% 		    		\glsprintmitdeutsch%
% 		    	}{%
% 			    	\ifglshasfield{englisch}{gls-\glslabel}{%
% 			    		\glsprintmitenglisch%
% 			    	}{%
% 			    		{\glsgenentryfmt}%
% 			    	}%
% 		    	}%
% 		    }%
% 		}%
% 		\newcommand*{\glsprintohnedeutsch}{%
% 		  \ifglsused{\glslabel}%
% 		    {\glsgenentryfmt}% Wenn verwendet normales format
% 		    {\glsgenentryfmt}% beim ersten mal anders formatiert
% 		}%
% 		% Alternativen für erste Formatierung:
% 		% textsf, textit, texttt
% 		%
% 		\defglsentryfmt[\acronymtype]{%
% 			\ifglsentryexists{gls-\glslabel}{%
% 				\glsprintmitdualentry%
% 			}{%
% 				\glsprintohnedeutsch%
% 			}%
% 		}%
% 		%
% 		\renewcommand*{\glsclearpage}{}%
% 		%
% 		\newlist{myglossarylist}{description}{10}%
% 		\setlist[myglossarylist]{%
% 			align=left,%
% 		% 	itemindent=0em,%
% 		% 	labelindent=0em,%
% 		% 	listparindent=0em,%
% 		}%
% 		\setlist[myglossarylist,2]{%
% 			align=left,%
% 		 	leftmargin=2em,%
% 			itemindent=-2em,%
% 		}%
% 		\newglossarystyle{list+url}
% 		{% based on list style (adapt as required)
% 		  \setglossarystyle{list}%
% 		    \renewcommand{\glossentry}[2]{%
% 		    \item[\glsentryitem{##1}%
% 		          \glstarget{##1}{\glossentryname{##1}}]
% 		       \glossentrydesc{##1}\glspostdescription\space##2%
% 		    \ifglshasfield{url}{##1}{\glspar
% 		     \glsletentryfield{\thisurl}{##1}{url}%
% 		     \expandafter\url\expandafter{\thisurl}}{}}%
% 		}%
% 		\newglossarystyle{list+Deutsch+Untertitel+alternativ+url}{%
% 			\renewenvironment{theglossary}%
% 				{\begin{myglossarylist}}{\end{myglossarylist}}%
% 			\renewcommand*{\glossaryheader}{}%
% 			\renewcommand*{\glsgroupheading}[1]{}%
% 			\renewcommand*{\glossentry}[2]{%
% 				\item[\glsentryitem{##1}%
% 					  \glstarget{##1}{\glossentryname{##1}}]%
% 		%
% 				\ifglshasfield{untertitel}{##1}{\normalfont%
% 					\textbf{(}\glsentryuntertitel{##1}\textbf{)}.\space}{}%
% 				\ifglshasfield{alternativ}{##1}{\normalfont%
% 					Auch genannt: \textbf{\glsentryalternativ{##1}}.\space}{}%
% 				\ifglshasfield{deutsch}{##1}{Zu Deutsch: %
% 					\textbf{\glsentrydeutsch{##1}}.\space}{}%
% 				\glossentrydesc{##1}\glspostdescription\space ##2%
% 		    \ifglshasfield{url}{##1}{\glspar
% 				\url{\glsentryurl{##1}}}{}%
% 			}%
% 			\renewcommand*{\subglossentry}[3]{%
% 				\begin{myglossarylist}%
% 					\item[\glssubentryitem{##2}%
% 					\glstarget{##2}{\strut\glossentryname{##2}%
% 					}]%
% 		%
% 					\ifglshasfield{untertitel}{##2}{\normalfont%
% 						\textbf{(}\glsentryuntertitel{##2}\textbf{)}.\space}{}%
% 					\ifglshasfield{alternativ}{##2}{\normalfont%
% 						Auch genannt: \textbf{\glsentryalternativ{##2}}.\space}{}%
% 					\ifglshasfield{deutsch}{##2}{Zu Deutsch: %
% 						\textbf{\glsentrydeutsch{##2}}.\space}{}%
% 					\glossentrydesc{##2}\glspostdescription\space ##3.%
% 				\end{myglossarylist}%
% 			}%
% 			\renewcommand*{\glsgroupskip}{\ifglsnogroupskip\else\indexspace\fi}%
% 		}%
% 		\setglossarystyle{list}%
%
% \setacronymstyle{long-short}
%/\/\/\/\/\/\/\/\/\/\/\/\/\/\/\/\/\/\/\/\/\/\/\/\/\/\/\/\/\/\/\/\/\/\/\/\
%							Glossaries fertig
%------------------------------------------------------------------------
%________________________________________________________________________
%
%
%
%
%________________________________________________________________________
%------------------------------------------------------------------------
%							Setups für enumitem
%					Listen - (itemize, enumerate, description)
%/\/\/\/\/\/\/\/\/\/\/\/\/\/\/\/\/\/\/\/\/\/\/\/\/\/\/\/\/\/\/\/\/\/\/\/\
% \setlist[1]{\labelindent=\parindent} % < Usually a good idea
\setlist[description]{font=\sffamily\bfseries} % the Default
%/\/\/\/\/\/\/\/\/\/\/\/\/\/\/\/\/\/\/\/\/\/\/\/\/\/\/\/\/\/\/\/\/\/\/\/\
%							enumitem fertig
%------------------------------------------------------------------------
%________________________________________________________________________
%
%
%
%
%
%
%________________________________________________________________________
%------------------------------------------------------------------------
%						Setups für Hyperref/URL
%		Note: Package hyperref internally laods package url
%		(Like some other Packages do. For Example: biblatex)
%/\/\/\/\/\/\/\/\/\/\/\/\/\/\/\/\/\/\/\/\/\/\/\/\/\/\/\/\/\/\/\/\/\/\/\/\
\mathchardef\UrlBreakPenalty=100% I think default is 100
% Allow linebreaks in URLs after additional character to the defaults, but
% don't just \renew or \def it, because this would remove all characters
% predefined within \UrlBreaks in the package. This could mislead other users.
% E.g. can cause smaller dots.. So use the etoolbox with "append to"
\appto\UrlBreaks{%
	\do\*%
% 	\do\-% Is done with the url option "hyphens"
	\do\~%
	\do\"%
	\do\a\do\b\do\c\do\d\do\e\do\f\do\g\do\h\do\i\do\j%
	\do\k\do\l\do\m\do\n\do\o\do\p\do\q\do\r\do\s\do\t\do\u\do\v\do\w%
	\do\x\do\y\do\z%
}%
%	Note: Alread inside the \UrlBreaks by default are:
% 	\do\.%
% 	\do\=%
% 	\do\'%
% 	\do\&%
% 	\do\-% With the option hypens
% \expandafter\def\expandafter\UrlBreaks\expandafter{\UrlBreaks% save the current one
% 	\do\*%
% 	\do\-%
% 	\do\~%
% 	\do\'%
% 	\do\"%
% 	\do\-%
% 	\do\&%
% 	\do\=%
% }%
%/\/\/\/\/\/\/\/\/\/\/\/\/\/\/\/\/\/\/\/\/\/\/\/\/\/\/\/\/\/\/\/\/\/\/\/\
%						Hyperref/URL fertig
%------------------------------------------------------------------------
%________________________________________________________________________
%
%
%
%
%
%
%
%________________________________________________________________________
%------------------------------------------------------------------------
%						Redefinition of \texttt
%				to allow a proper hyphenation (Silbentrennung)
% other breakpoint symbols can be added, for example, below I also make [ a breakpoint:
%/\/\/\/\/\/\/\/\/\/\/\/\/\/\/\/\/\/\/\/\/\/\/\/\/\/\/\/\/\/\/\/\/\/\/\/\
\let\stdtexttt\texttt
% \newcommand{\newtexttt}[1]{%
%   \begingroup
%   \ttfamily
%   \begingroup\lccode`~=`/\lowercase{\endgroup\def~}{/\discretionary{}{}{}}%
%   \begingroup\lccode`~=`[\lowercase{\endgroup\def~}{[\discretionary{}{}{}}%
%   \begingroup\lccode`~=`.\lowercase{\endgroup\def~}{.\discretionary{}{}{}}%
%   \catcode`/=\active\catcode`[=\active\catcode`.=\active
%   \scantokens{#1\noexpand}%
%   \endgroup
% }
% Oh noes, better way: Look at the Schriftart Segment in the Header
%/\/\/\/\/\/\/\/\/\/\/\/\/\/\/\/\/\/\/\/\/\/\/\/\/\/\/\/\/\/\/\/\/\/\/\/\
%						Redef fertig
%------------------------------------------------------------------------
%________________________________________________________________________
%
%
%
%
%
%________________________________________________________________________
%------------------------------------------------------------------------
%							Setups für Cleveref
%/\/\/\/\/\/\/\/\/\/\/\/\/\/\/\/\/\/\/\/\/\/\/\/\/\/\/\/\/\/\/\/\/\/\/\/\
\crefname{chapter}{Kapitel}{Kapiteln}
\crefname{section}{Abschnitt}{Abschnitten}
\crefname{part}{Teil}{Teilen}
\crefname{figure}{Abbildung}{Abbildungen}
\crefname{table}{Tabelle}{Tabellen}
\crefname{listing}{Listing}{Listings}
\crefname{equation}{Formel}{Formeln}
%/\/\/\/\/\/\/\/\/\/\/\/\/\/\/\/\/\/\/\/\/\/\/\/\/\/\/\/\/\/\/\/\/\/\/\/\
%							Cleveref fertig
%------------------------------------------------------------------------
%________________________________________________________________________
%
%
%
%
%
%________________________________________________________________________
%------------------------------------------------------------------------
%						Definitionen allgemeiner Farben
%/\/\/\/\/\/\/\/\/\/\/\/\/\/\/\/\/\/\/\/\/\/\/\/\/\/\/\/\/\/\/\/\/\/\/\/\
\definecolor{optionalargscolor}{RGB}{255,125,25}%
\definecolor{ColorRegExCmdID}{RGB}{25,0,255}%
\definecolor{ColorRegExCmdSelect}{RGB}{255,125,25}%
\definecolor{ColorRegExCmdMultiSelect}{RGB}{30,150,0}%
\definecolor{ColorRegExCmdOption}{RGB}{100,100,100}%
%/\/\/\/\/\/\/\/\/\/\/\/\/\/\/\/\/\/\/\/\/\/\/\/\/\/\/\/\/\/\/\/\/\/\/\/\
%							Farben fertig
%------------------------------------------------------------------------
%________________________________________________________________________
%
%
%
%
%
%
%
%
% Initialisierungen wie Counter, Theorem, Variablen, Konstanten..
%
\theoremstyle{break}
\theoremheaderfont{\bfseries}
\theorembodyfont{\normalfont}
\theoremseparator{}
\theorempreskip{0ex}
\theorempostskip{0ex}
\theoremindent0.5em
%
%
% 
% 
%
%
%
%
\input{./organization/settings_listings.tex}%%
\microtypesetup{activate=false}%
%%%%%%%%%%%%%%%%%%%%%%%%%%%%%%%%%%%%%%%%%%%%%%%%%%%%%%%%%%%%%%%%%
% Tracking, Spacing & Kerning werden seit Lua(La)Tex von 'fontspec' verwaltet
% 				% \microtypesetup{tracking=false}%
% 				% \microtypesetup{kerning=false}%
% 				% \microtypesetup{spacing=false}%
%
%================================================================
%----------------------------------------------------------------
%      Title Page \& Schmutztitel
%================================================================
%
%
% Standard-Title-Page Daten. Actually not used...
%
\title{}%
\author{DenKr}%
\date{\copyright \today}%
%
%
%
%
% Titlepage - Daten für eigene Title Page
%
\newcommand{\TPtitle}{%
Example\nl%
例\nl%
\begin{ocg}{Romaji TitlePage}{romaji-titlePage}{0}%
れい
rei%
\par\end{ocg}%
\switchocg{romaji-titlePage}{\fbox{\tiny ローマ字}}%
}%
\newcommand{\TPpublicationType}{\vspace{60pt}}%
\newcommand{\TPauthorAnrede}{}%
\newcommand{\TPauthorTitleBefore}{}%
\newcommand{\TPauthorTitleAfter}{M.Sc.}%
\newcommand{\TPauthorFirst}{Den}%
\newcommand{\TPauthorLast}{Kr}%
\newcommand{\TPauthorStreet}{}%
\newcommand{\TPauthorPLZ}{}%
\newcommand{\TPauthorCity}{}%
\newcommand{\TPauthorMail}{}%
\newcommand{\TPsignedAtCity}{}%
\newcommand{\TPsupervisorOneAnrede}{}%
\newcommand{\TPsupervisorOneTitleBefore}{}%
\newcommand{\TPsupervisorOneTitleAfter}{}%
\newcommand{\TPsupervisorOneFirst}{}%
\newcommand{\TPsupervisorOneLast}{}%
\newcommand{\TPsupervisorTwoAnrede}{}%
\newcommand{\TPsupervisorTwoTitleBefore}{}%
\newcommand{\TPsupervisorTwoTitleAfter}{}%
\newcommand{\TPsupervisorTwoFirst}{}%
\newcommand{\TPsupervisorTwoLast}{}%
\newcommand{\TPsupervisorThreeAnrede}{}%
\newcommand{\TPsupervisorThreeTitleBefore}{}%
\newcommand{\TPsupervisorThreeTitleAfter}{}%
\newcommand{\TPsupervisorThreeFirst}{A}%
\newcommand{\TPsupervisorThreeLast}{B}%
% \newcommand{\TPdate}{\today}%
\newcommand{\TPdate}{\today}%
%%
%\maketitle
%
%
%	Usage Notes:
% 	\ifdefempty needs the \usepackage{etoolbox}
%
%
%
% 
% 
% 
%
\newcommand{\HRule}{\rule{\linewidth}{0.5mm}}%
%
%
\begin{titlepage}%
%
\begin{center}%
%
%
% Oberer Teil der Titelseite:
\begin{minipage}{0.4\textwidth}%
%\includegraphics[height=0.15\textheight]{./bilder/logo.png}\\[1cm]
\begin{flushleft}%
% \includegraphics[height=4\baselineskip]{./graphics/core/Logos/Some_Logo.png}%
% \textsc{\Large Affiliation}\\
% \textsc{\Large }%
\end{flushleft}%
\end{minipage}%
\hfill%
\begin{minipage}{0.5\textwidth}%
\begin{flushright}%
% {\Large \fontfamily{phv}\fontseries{m}\fontshape{n}\selectfont Affil Part1}\\%
% {\Large \fontfamily{phv}\fontseries{m}\fontshape{n}\selectfont Affil Part2}\\%
% {\Large \fontfamily{phv}\fontseries{m}\fontshape{n}\selectfont Affil Part3}\\%
% {\Large \fontfamily{phv}\fontseries{m}\fontshape{n}\selectfont Affil Part4}%
\end{flushright}%
\end{minipage}%
%
\vspace{0.07\textheight}%
%
\begin{spacing}{1.2}%
\textsc{\LARGE\TPpublicationType}%
\end{spacing}%
%
\vspace{0.02\textheight}%
%
%
% Title
\HRule \\[0.2\baselineskip]%
\begin{spacing}{1}%
\huge\bfseries%
\TPtitle%
\\[0\baselineskip]%
\end{spacing}%
\HRule \\[0.08\textheight]%
%
% Author and supervisor
\begin{minipage}[t]{0.47\textwidth}%
\begin{flushleft} \large%
{Autor:}\\%
\ifdefempty{\TPauthorAnrede}%
	{}%
	{\TPauthorAnrede\ }%
\ifdefempty{\TPauthorTitleBefore}%
	{}%
	{\TPauthorTitleBefore\ }%
\TPauthorFirst\ \textsc{\TPauthorLast}%
\ifdefempty{\TPauthorTitleAfter}%
	{}%
	{, \TPauthorTitleAfter}%
\end{flushleft}%
\end{minipage}%
\hfill%
\begin{minipage}[t]{0.52\textwidth}%
\begin{flushright} \large%
% \vspace*{\baselineskip}%
% {Betreuer:}\\%
{\ }\\%
\ifdefempty{\TPsupervisorOneAnrede}%
	{}%
	{\TPsupervisorOneAnrede\ }%
\ifdefempty{\TPsupervisorOneTitleBefore}%
	{}%
	{\TPsupervisorOneTitleBefore\ }%
\TPsupervisorOneFirst\ \textsc{\TPsupervisorOneLast}%
\ifdefempty{\TPsupervisorOneTitleAfter}%
	{}%
	{, \TPsupervisorOneTitleAfter}%
\\%
\ifdefempty{\TPsupervisorTwoAnrede}%
	{}%
	{\TPsupervisorTwoAnrede\ }%
\ifdefempty{\TPsupervisorTwoTitleBefore}%
	{}%
	{\TPsupervisorTwoTitleBefore\ }%
\TPsupervisorTwoFirst\ \textsc{\TPsupervisorTwoLast}%
\ifdefempty{\TPsupervisorTwoTitleAfter}%
	{}%
	{, \TPsupervisorTwoTitleAfter}%
\\%
\end{flushright}%
\end{minipage}%
%
\vfill%
%
% Unterer Teil der Seite
{\large \TPdate}%
%
\end{center}%
%
\end{titlepage}%%
%
% Title Page Rückseite
%
\vspace*{\fill}%
\begin{flushright}%
\begin{large}%
\TPauthorFirst\ \TPauthorLast\nl%
% v0.1 - 02.09.2014\nl
% v1.0 - 03.11.2014\nl
% v2.0 - 28.01.2015\nl
% Final - 28.01.2015\nl
% aktuell: \today
\TPdate%
\end{large}%
\end{flushright}%
\clearpage%
%
%================================================================
%----------------------------------------------------------------
%      Inhaltsverzeichnis
%================================================================
%
\begingroup%
  \renewcommand*{\chapterpagestyle}{empty}%
  \pagestyle{plain}%
  \tableofcontents%
%   \thispagestyle{empty}%
  \clearpage%
%   \listoffigures%
%   \clearpage%
%   \listoftables%
%   \clearpage%
\endgroup%
%
%================================================================
%----------------------------------------------------------------
%      Setup, Typographisch
%             (Für eigentliches Dokument)
%================================================================
%
% \selectlanguage{ngerman}%
%
%\sloppy % Würde die Dehnung der Zwischenräume etwas lockerer machen
%\fussy % Schaltet \sloppy wieder aus. Ist der Latex-Standart
\microtypesetup{activate=true}%
%%%%%%%%%%%%%%%%%%%%%%%%%%%%%%%%%%%%%%%%%%%%%%%%%%%%%%%%%%%%%%%%%
% Tracking, Spacing & Kerning werden seit Lua(La)Tex von 'fontspec' verwaltet
% 					% \microtypesetup{tracking=true}%
% 					% \microtypesetup{kerning=true}%
% 					% \microtypesetup{spacing=true}%
%
% Beachte: Typographisch erhalten Kapitelseiten keinen Kolumnentitel: plain
\renewcommand*{\chapterpagestyle}{plain.scrheadings}%
\pagestyle{scrheadings}%
\renewcommand*{\partpagestyle}{empty}%
%
%
%
%
%
%________________________________________________________________________
%------------------------------------------------------------------------
%		Spacing für
%					- Equation Environment
%					- floats
%					- Some new Lengths & Variables
%/\/\/\/\/\/\/\/\/\/\/\/\/\/\/\/\/\/\/\/\/\/\/\/\/\/\/\/\/\/\/\/\/\/\/\/\
% - - - - - - - - - - - - - - - - - - - - - - - - - - - - - - - - - - - -
%- - - - - - - - - - - - - - - - - - - - - - - - - - - - - - - - - - - - -
% - - - - - - - - - - - - - - - - - - - - - - - - - - - - - - - - - - - -
%		Equation
% - - - - - - - - - - - - - - - - - - - - - - - - - - - - - - - - - - - -
%------------------------------------------------------------------------
% 	\begin{equation}\begin{split}
% 	MRes_{K}^{3rd} = \delta_{3rd} + \delta_{K}
% 	\end{split}\end{equation}
%------------------------------------------------------------------------
% Default for 'above' & 'below' Space:
%		11.0pt plus 3.0pt minus 6.0pt
% You can check this using (prints it into the Document):
%		\the\abovedisplayskip
%		\the\belowdisplayskip
%------------------------------------------------------------------------
% The ones with 'short' come in play if the last line immediately before an equation is, well, short.
%/\/\/\/\/\/\/\/\/\/\/\/\/\/\/\/\/\/\/\/\/\/\/\/\/\/\/\/\/\/\/\/\/\/\/\/\
\setlength{\abovedisplayskip}{3pt}
\setlength{\belowdisplayskip}{3pt}
\setlength{\abovedisplayshortskip}{0pt}
\setlength{\belowdisplayshortskip}{0pt}
% - - - - - - - - - - - - - - - - - - - - - - - - - - - - - - - - - - - -
%- - - - - - - - - - - - - - - - - - - - - - - - - - - - - - - - - - - - -
% - - - - - - - - - - - - - - - - - - - - - - - - - - - - - - - - - - - -
%		Floats
% - - - - - - - - - - - - - - - - - - - - - - - - - - - - - - - - - - - -
\setlength\extrarowheight{1ex}%
% The Spacing between floats and the surrounding text at top and bottom of the
% float. The Defaults are somewhat complicated and depend on document type and
% font-size%
% \setlength{\intextsep}{\intextsep}%
% Some more lenghts like this
%     \textfloatsep — distance between floats on the top or the bottom and the text;
%     \floatsep — distance between two floats;
%     \intextsep — distance between floats inserted inside the page text (using h) and the text proper.
% - - - - - - - - - - - - - - - - - - - - - - - - - - - - - - - - - - - -
%- - - - - - - - - - - - - - - - - - - - - - - - - - - - - - - - - - - - -
% - - - - - - - - - - - - - - - - - - - - - - - - - - - - - - - - - - - -
%		New Lenghts & Variables
% - - - - - - - - - - - - - - - - - - - - - - - - - - - - - - - - - - - -
\newlength\textheighttemp%
% \setlength{\textheighttemp}{\textheight}%
\newlength\textwidthtemp%
% \setlength{\textwidthtemp}{\textwidth}%
\newlength\textheightstd%
\setlength{\textheightstd}{\textheight}%
\newlength\textwidthstd%
\setlength{\textwidthstd}{\textwidth}%
\newlength\textheightold%
\newlength\textwidthold%
%
\newlength\tempheight%
\newlength\tempwidth%
%\/\/\/\/\/\/\/\/\/\/\/\/\/\/\/\/\/\/\/\/\/\/\/\/\/\/\/\/\/\/\/\/\/\/\/\/
%			Spacing End
%------------------------------------------------------------------------
%________________________________________________________________________
%
% ( '\setlength' must come after \begin{document} )
%/\/\/\/\/\/\/\/\/\/\/\/\/\/\/\/\/\/\/\/\/\/\/\/\/\/\/\/\/\/\/\/\/\/\/\/\
%			Spacing End
%------------------------------------------------------------------------
%________________________________________________________________________
%
%
%
%
%
%
%
%
%
%________________________________________________________________________
%------------------------------------------------------------------------
%							Setups für Microtype
%/\/\/\/\/\/\/\/\/\/\/\/\/\/\/\/\/\/\/\/\/\/\/\/\/\/\/\/\/\/\/\/\/\/\/\/\
% \SetProtrusion{encoding={*},family={bch},series={*},size={6,7}}
%               {1={ ,750},2={ ,500},3={ ,500},4={ ,500},5={ ,500},
%                6={ ,500},7={ ,600},8={ ,500},9={ ,500},0={ ,500}}
%%%%%%%%%%%%%%%%%%%%%%%%%%%%%%%%%%%%%%%%%%%%%%%%%%%%%%%%%%%%%%%%%
% Tracking, Spacing & Kerning werden seit Lua(La)Tex von 'fontspec' verwaltet
% 				% \SetExtraKerning[unit=space]
% 				%     {encoding={*}, family={bch}, series={*}, size={footnotesize,small,normalsize}}
% 				%     {\textendash={400,400}, % en-dash, add more space around it
% 				%      "28={ ,150}, % left bracket, add space from right
% 				%      "29={150, }, % right bracket, add space from left
% 				%      \textquotedblleft={ ,150}, % left quotation mark, space from right
% 				%      \textquotedblright={150, }} % right quotation mark, space from left
% 				% \SetTracking{encoding={*}, shape=sc}{40}
%\/\/\/\/\/\/\/\/\/\/\/\/\/\/\/\/\/\/\/\/\/\/\/\/\/\/\/\/\/\/\/\/\/\/\/\/
%							Microtype fertig
%------------------------------------------------------------------------
%________________________________________________________________________
%
%
%
%
%
%
%
%________________________________________________________________________
%------------------------------------------------------------------------
%							Setups für Glossaries
%/\/\/\/\/\/\/\/\/\/\/\/\/\/\/\/\/\/\/\/\/\/\/\/\/\/\/\/\/\/\/\/\/\/\/\/\
%Formatierung für Abkürzungen bei erstem Auftreten und weiteren und
%Falls deutsch Key im GLossar vorhanden, benutze ihn beim ersten Mal
% \defglsentryfmt[\acronymtype]{
%   \ifglsused{\glslabel}
%     {\glsgenentryfmt}% Wenn verwendet normales format
%     {\textit{\glsgenentryfmt}}% beim ersten mal kursiv
% }
% 		\newcommand*{\glsprintmitdeutsch}{%
% 			{\glsentrylong{\glslabel} (\glsentryname{\glslabel}, dt.:
% 			\glsentrydeutsch{gls-\glslabel})}%
% 		}%
% 		\newcommand*{\glsprintmitenglisch}{%
% 			{\glsentrylong{\glslabel} (\glsentryname{\glslabel}, engl.:
% 			\glsentryenglisch{gls-\glslabel})}%
% 		}%
% 		\newcommand*{\glsprintmitdualentry}{%
% 		  \ifglsused{\glslabel}%
% 		    {%
% 		    	\glsgenentryfmt%
% 		    }{%
% 		    	\ifglshasfield{deutsch}{gls-\glslabel}{%
% 		    		\glsprintmitdeutsch%
% 		    	}{%
% 			    	\ifglshasfield{englisch}{gls-\glslabel}{%
% 			    		\glsprintmitenglisch%
% 			    	}{%
% 			    		{\glsgenentryfmt}%
% 			    	}%
% 		    	}%
% 		    }%
% 		}%
% 		\newcommand*{\glsprintohnedeutsch}{%
% 		  \ifglsused{\glslabel}%
% 		    {\glsgenentryfmt}% Wenn verwendet normales format
% 		    {\glsgenentryfmt}% beim ersten mal anders formatiert
% 		}%
% 		% Alternativen für erste Formatierung:
% 		% textsf, textit, texttt
% 		%
% 		\defglsentryfmt[\acronymtype]{%
% 			\ifglsentryexists{gls-\glslabel}{%
% 				\glsprintmitdualentry%
% 			}{%
% 				\glsprintohnedeutsch%
% 			}%
% 		}%
% 		%
% 		\renewcommand*{\glsclearpage}{}%
% 		%
% 		\newlist{myglossarylist}{description}{10}%
% 		\setlist[myglossarylist]{%
% 			align=left,%
% 		% 	itemindent=0em,%
% 		% 	labelindent=0em,%
% 		% 	listparindent=0em,%
% 		}%
% 		\setlist[myglossarylist,2]{%
% 			align=left,%
% 		 	leftmargin=2em,%
% 			itemindent=-2em,%
% 		}%
% 		\newglossarystyle{list+url}
% 		{% based on list style (adapt as required)
% 		  \setglossarystyle{list}%
% 		    \renewcommand{\glossentry}[2]{%
% 		    \item[\glsentryitem{##1}%
% 		          \glstarget{##1}{\glossentryname{##1}}]
% 		       \glossentrydesc{##1}\glspostdescription\space##2%
% 		    \ifglshasfield{url}{##1}{\glspar
% 		     \glsletentryfield{\thisurl}{##1}{url}%
% 		     \expandafter\url\expandafter{\thisurl}}{}}%
% 		}%
% 		\newglossarystyle{list+Deutsch+Untertitel+alternativ+url}{%
% 			\renewenvironment{theglossary}%
% 				{\begin{myglossarylist}}{\end{myglossarylist}}%
% 			\renewcommand*{\glossaryheader}{}%
% 			\renewcommand*{\glsgroupheading}[1]{}%
% 			\renewcommand*{\glossentry}[2]{%
% 				\item[\glsentryitem{##1}%
% 					  \glstarget{##1}{\glossentryname{##1}}]%
% 		%
% 				\ifglshasfield{untertitel}{##1}{\normalfont%
% 					\textbf{(}\glsentryuntertitel{##1}\textbf{)}.\space}{}%
% 				\ifglshasfield{alternativ}{##1}{\normalfont%
% 					Auch genannt: \textbf{\glsentryalternativ{##1}}.\space}{}%
% 				\ifglshasfield{deutsch}{##1}{Zu Deutsch: %
% 					\textbf{\glsentrydeutsch{##1}}.\space}{}%
% 				\glossentrydesc{##1}\glspostdescription\space ##2%
% 		    \ifglshasfield{url}{##1}{\glspar
% 				\url{\glsentryurl{##1}}}{}%
% 			}%
% 			\renewcommand*{\subglossentry}[3]{%
% 				\begin{myglossarylist}%
% 					\item[\glssubentryitem{##2}%
% 					\glstarget{##2}{\strut\glossentryname{##2}%
% 					}]%
% 		%
% 					\ifglshasfield{untertitel}{##2}{\normalfont%
% 						\textbf{(}\glsentryuntertitel{##2}\textbf{)}.\space}{}%
% 					\ifglshasfield{alternativ}{##2}{\normalfont%
% 						Auch genannt: \textbf{\glsentryalternativ{##2}}.\space}{}%
% 					\ifglshasfield{deutsch}{##2}{Zu Deutsch: %
% 						\textbf{\glsentrydeutsch{##2}}.\space}{}%
% 					\glossentrydesc{##2}\glspostdescription\space ##3.%
% 				\end{myglossarylist}%
% 			}%
% 			\renewcommand*{\glsgroupskip}{\ifglsnogroupskip\else\indexspace\fi}%
% 		}%
% 		\setglossarystyle{list}%
%
% \setacronymstyle{long-short}
%/\/\/\/\/\/\/\/\/\/\/\/\/\/\/\/\/\/\/\/\/\/\/\/\/\/\/\/\/\/\/\/\/\/\/\/\
%							Glossaries fertig
%------------------------------------------------------------------------
%________________________________________________________________________
%
%
%
%
%________________________________________________________________________
%------------------------------------------------------------------------
%							Setups für enumitem
%					Listen - (itemize, enumerate, description)
%/\/\/\/\/\/\/\/\/\/\/\/\/\/\/\/\/\/\/\/\/\/\/\/\/\/\/\/\/\/\/\/\/\/\/\/\
% \setlist[1]{\labelindent=\parindent} % < Usually a good idea
\setlist[description]{font=\sffamily\bfseries} % the Default
%/\/\/\/\/\/\/\/\/\/\/\/\/\/\/\/\/\/\/\/\/\/\/\/\/\/\/\/\/\/\/\/\/\/\/\/\
%							enumitem fertig
%------------------------------------------------------------------------
%________________________________________________________________________
%
%
%
%
%
%
%________________________________________________________________________
%------------------------------------------------------------------------
%						Setups für Hyperref/URL
%		Note: Package hyperref internally laods package url
%		(Like some other Packages do. For Example: biblatex)
%/\/\/\/\/\/\/\/\/\/\/\/\/\/\/\/\/\/\/\/\/\/\/\/\/\/\/\/\/\/\/\/\/\/\/\/\
\mathchardef\UrlBreakPenalty=100% I think default is 100
% Allow linebreaks in URLs after additional character to the defaults, but
% don't just \renew or \def it, because this would remove all characters
% predefined within \UrlBreaks in the package. This could mislead other users.
% E.g. can cause smaller dots.. So use the etoolbox with "append to"
\appto\UrlBreaks{%
	\do\*%
% 	\do\-% Is done with the url option "hyphens"
	\do\~%
	\do\"%
	\do\a\do\b\do\c\do\d\do\e\do\f\do\g\do\h\do\i\do\j%
	\do\k\do\l\do\m\do\n\do\o\do\p\do\q\do\r\do\s\do\t\do\u\do\v\do\w%
	\do\x\do\y\do\z%
}%
%	Note: Alread inside the \UrlBreaks by default are:
% 	\do\.%
% 	\do\=%
% 	\do\'%
% 	\do\&%
% 	\do\-% With the option hypens
% \expandafter\def\expandafter\UrlBreaks\expandafter{\UrlBreaks% save the current one
% 	\do\*%
% 	\do\-%
% 	\do\~%
% 	\do\'%
% 	\do\"%
% 	\do\-%
% 	\do\&%
% 	\do\=%
% }%
%/\/\/\/\/\/\/\/\/\/\/\/\/\/\/\/\/\/\/\/\/\/\/\/\/\/\/\/\/\/\/\/\/\/\/\/\
%						Hyperref/URL fertig
%------------------------------------------------------------------------
%________________________________________________________________________
%
%
%
%
%
%
%
%________________________________________________________________________
%------------------------------------------------------------------------
%						Redefinition of \texttt
%				to allow a proper hyphenation (Silbentrennung)
% other breakpoint symbols can be added, for example, below I also make [ a breakpoint:
%/\/\/\/\/\/\/\/\/\/\/\/\/\/\/\/\/\/\/\/\/\/\/\/\/\/\/\/\/\/\/\/\/\/\/\/\
\let\stdtexttt\texttt
% \newcommand{\newtexttt}[1]{%
%   \begingroup
%   \ttfamily
%   \begingroup\lccode`~=`/\lowercase{\endgroup\def~}{/\discretionary{}{}{}}%
%   \begingroup\lccode`~=`[\lowercase{\endgroup\def~}{[\discretionary{}{}{}}%
%   \begingroup\lccode`~=`.\lowercase{\endgroup\def~}{.\discretionary{}{}{}}%
%   \catcode`/=\active\catcode`[=\active\catcode`.=\active
%   \scantokens{#1\noexpand}%
%   \endgroup
% }
% Oh noes, better way: Look at the Schriftart Segment in the Header
%/\/\/\/\/\/\/\/\/\/\/\/\/\/\/\/\/\/\/\/\/\/\/\/\/\/\/\/\/\/\/\/\/\/\/\/\
%						Redef fertig
%------------------------------------------------------------------------
%________________________________________________________________________
%
%
%
%
%
%________________________________________________________________________
%------------------------------------------------------------------------
%							Setups für Cleveref
%/\/\/\/\/\/\/\/\/\/\/\/\/\/\/\/\/\/\/\/\/\/\/\/\/\/\/\/\/\/\/\/\/\/\/\/\
\crefname{chapter}{Kapitel}{Kapiteln}
\crefname{section}{Abschnitt}{Abschnitten}
\crefname{part}{Teil}{Teilen}
\crefname{figure}{Abbildung}{Abbildungen}
\crefname{table}{Tabelle}{Tabellen}
\crefname{listing}{Listing}{Listings}
\crefname{equation}{Formel}{Formeln}
%/\/\/\/\/\/\/\/\/\/\/\/\/\/\/\/\/\/\/\/\/\/\/\/\/\/\/\/\/\/\/\/\/\/\/\/\
%							Cleveref fertig
%------------------------------------------------------------------------
%________________________________________________________________________
%
%
%
%
%
%________________________________________________________________________
%------------------------------------------------------------------------
%						Definitionen allgemeiner Farben
%/\/\/\/\/\/\/\/\/\/\/\/\/\/\/\/\/\/\/\/\/\/\/\/\/\/\/\/\/\/\/\/\/\/\/\/\
\definecolor{optionalargscolor}{RGB}{255,125,25}%
\definecolor{ColorRegExCmdID}{RGB}{25,0,255}%
\definecolor{ColorRegExCmdSelect}{RGB}{255,125,25}%
\definecolor{ColorRegExCmdMultiSelect}{RGB}{30,150,0}%
\definecolor{ColorRegExCmdOption}{RGB}{100,100,100}%
%/\/\/\/\/\/\/\/\/\/\/\/\/\/\/\/\/\/\/\/\/\/\/\/\/\/\/\/\/\/\/\/\/\/\/\/\
%							Farben fertig
%------------------------------------------------------------------------
%________________________________________________________________________
%
%
%
%
%
%
%
%
% Initialisierungen wie Counter, Theorem, Variablen, Konstanten..
%
\theoremstyle{break}
\theoremheaderfont{\bfseries}
\theorembodyfont{\normalfont}
\theoremseparator{}
\theorempreskip{0ex}
\theorempostskip{0ex}
\theoremindent0.5em
%
%
% 
% 
%
%
%
%
%________________________________________________________________________
%------------------------------------------------------------------------
%							listings - Setup. Source-Code Einbindung.
%/\/\/\/\/\/\/\/\/\/\/\/\/\/\/\/\/\/\/\/\/\/\/\/\/\/\/\/\/\/\/\/\/\/\/\/\
\newcommand{\lstC}[1]{\lstinline[language=C,breaklines=true,basicstyle=\listingsfontinline]$#1$}
% Colors etc. to get the Eclipse Look
% \newfontfamily\listingsfont[Scale=0.7]{Courier} 
% \newfontfamily\listingsfontinline[Scale=0.8]{Courier New} 
\definecolor{eclipse_comment}{rgb}{0.12, 0.38, 0.18 } %adjusted, in Eclipse: {0.25, 0.42, 0.30 } = #3F6A4D
\definecolor{eclipse_keyword1}{RGB}{128, 0, 40}  % red
\definecolor{eclipse_keyword2}{RGB}{155,40,128} %purple
\definecolor{eclipse_keyword3}{RGB}{0,120,10} %green
\definecolor{eclipse_string}{rgb}{0.06, 0.10, 0.98} % #101AF9
\def\lstsmallmath{\leavevmode\ifmmode \scriptstyle \else  \fi}
\def\lstsmallmathend{\leavevmode\ifmmode  \else  \fi}
%-----------
\definecolor{lstbg}{RGB}{230,230,230}
\lstset{language=C,texcl=true}
% Basic lstset (optical settings)
% \lstset{
% 	numbers=left,
% 	numberstyle=\tiny,
% 	numbersep=5pt,
% 	tabsize=4,
% 	xleftmargin=0em,
% 	backgroundcolor=\color{lstbg},
% 	keywordstyle=\bfseries\ttfamily\color{blue},
% 	stringstyle=\color{green}\ttfamily,
% 	commentstyle=\color{gray}\ttfamily,
% }
% lstset for Look like Eclipse
\lstset{
	frame=shadowbox,
	numbers=left,
	numberstyle=\tiny,
	numbersep=5pt,
	showspaces=false,
	showtabs=false,
	tabsize=2,
	breaklines=true,
	keepspaces=true,
% 	basicstyle=\listingsfont,
	captionpos=b,
	xleftmargin=0em,
	xrightmargin=0.3em,
	backgroundcolor=\color{lstbg},
	keywordstyle=\color{eclipse_keyword1}\bfseries,
	keywordstyle=[2]\color{eclipse_keyword2}\bfseries,
	keywordstyle=[3]\color{eclipse_keyword3}\bfseries,
	keywordstyle=[9]\color{red}\bfseries,
	stringstyle=\color{eclipse_string},
	commentstyle=\color{eclipse_comment}\itshape,
	escapebegin={\lstsmallmath}, escapeend={\lstsmallmathend}
}
\lstset{%
	inputencoding=utf8,%
	extendedchars=true,%
	literate=%
		{ß}{{\ss}}2%
		{ö}{{\"o}}1%
		{ä}{{\"a}}1%
		{ü}{{\"u}}1%
		{Ö}{{\"O}}1%
		{Ä}{{\"A}}1%
		{Ü}{{\"U}}1%
		{>>}{{\guillemotright}}1%
		{»}{{\guillemotright}}1%
		{<<}{{\guillemotleft}}1%
		{«}{{\guillemotleft}}1,%
}%
% Zusätzliche Hervorherbungen (for every language)
\lstset{
	emph={%  
    	critical,
    	section
    },
    emphstyle={\color{red}\bfseries}%\underbar}%
}%
\lstdefinelanguage{C_var}
{
	language=C,
	morecomment=[l]{//},
	morekeywords={
		strlen,
	},
	morekeywords=[2]{
		malloc,
		recv,
		send,
		sem_init,
		sem_wait,
		sem_post,
		sem_getvalue,
		sem_destroy,
		raise,
		signal,
		pthread_create,
		pthread_join,
		pthread_detach,
		pthread_cancel,
		pthread_setcancelstate,
		pcap_breakloop,
		pcap_close
	},
	morekeywords=[3]{
		int8_t,
		uint8_t,
		int16_t,
		uint16_t,
		int32_t,
		uint32_t,
		int64_t,
		uint64_t,
		int128_t,
		uint128_t,
		uintptr_t,
		sem_t,
		pthread_t
	},
	morekeywords=[9]{
		critical section
	},
}
%Additional languages with keywords
\lstdefinelanguage{semaphore}
{
	morekeywords={
		type,
		record,
		end,
		if,
		then,
		else,
		loop
	},
	morekeywords=[2]{
		wait,
		signal,
		await,
		advance,
		sem_wait,
		sem_post,
		sem_init,
		sem_getvalue
	},
	morekeywords=[3]{
		val
	},
	morekeywords=[4]{
		ticket
	},
	sensitive=true,
	morecomment=[l]{//},
	morecomment=[s]{/*}{*/},
	morestring=[b]",
	morestring=[b]',
	keywordstyle=\bfseries\ttfamily\color{blue},
	keywordstyle=[2]\bfseries\ttfamily\color{teal},
	keywordstyle=[3]\bfseries\ttfamily\color{olive},
	keywordstyle=[4]\bfseries\ttfamily\color{olive},
	stringstyle=\color{green}\ttfamily,
}
%/\/\/\/\/\/\/\/\/\/\/\/\/\/\/\/\/\/\/\/\/\/\/\/\/\/\/\/\/\/\/\/\/\/\/\/\
%							Listings fertig
%------------------------------------------------------------------------
%________________________________________________________________________%%
%
%
%