\usepackage[german,ngerman]{babel}%
\usepackage[utf8]{inputenc}%
\usepackage[T1]{fontenc}%
\usepackage[babel,german=quotes]{csquotes}%
%
%
%________________________________________________________________________
%------------------------------------------------------------------------
%							PassOptions Preparation
%	Die PassOptions sollten vor der ersten Inklusion des Packages erfolgen
%	Also auch bevor ein Package von einem anderen eingebunden wird...
%/\/\/\/\/\/\/\/\/\/\/\/\/\/\/\/\/\/\/\/\/\/\/\/\/\/\/\/\/\/\/\/\/\/\/\/\
\PassOptionsToPackage{usenames,dvipsnames,svgnames,x11names,table,prologue}{xcolor}%
\PassOptionsToPackage{hyphens}{url}%
\PassOptionsToPackage{nomessages}{fp}%
%/\/\/\/\/\/\/\/\/\/\/\/\/\/\/\/\/\/\/\/\/\/\/\/\/\/\/\/\/\/\/\/\/\/\/\/\
%							PassOptions fertig
%------------------------------------------------------------------------
%========================================================================
%
%
%
%________________________________________________________________________
%------------------------------------------------------------------------
%							Schriftarten einstellen
%/\/\/\/\/\/\/\/\/\/\/\/\/\/\/\/\/\/\/\/\/\/\/\/\/\/\/\/\/\/\/\/\/\/\/\/\
%\usepackage[scaled]{uarial}
% \usepackage{bookman}
\usepackage{lmodern}% Lädt Latin Modern Font und setzt es fürs Dokument
%Ist weniger Pixelig in pdf als Latex Standart
\usepackage{textcomp}% Lädt Text Companion Font
%Stellt insbesondere Zeichen zur verfügung, wie
%baht, bul­let, copy­right, mu­si­cal­note, onequar­ter, sec­tion, and yen
%------------------------------------------------
% Standartschriftart festlegen:
% Mögliche Werte
% \rmdefault - Roman (Serifen) Font
% \sfdefault - Sans Serif Font
% \ttdefault - TypeWriter Font
\renewcommand*{\familydefault}{\rmdefault}%
%Die drei Schriftfamilien einstellen:
\renewcommand*{\rmdefault}{lmr}%
% Standartmäßig verfügbare Schriftarten:
% cmr 	Computer Modern Roman (default)
% lmr 	Latin Modern Roman
% pbk 	Bookman
% bch 	Charter
% pnc 	New Century Schoolbook
% ppl 	Palatino
% ptm 	Times
\renewcommand*{\sfdefault}{lmss}%
% Standartmäßig verfügbare Schriftarten:
% cmss 	Computer Modern Sans Serif (default)
% lmss 	Latin Modern Sans Serif
% pag 	Avant Garde
% phv 	Helvetica
\renewcommand*{\ttdefault}{lmtt}%
% Standartmäßig verfügbare Schriftarten:
% cmtt 	Computer Modern Typewriter (default)
% lmtt 	Latin Modern
% pcr 	Courier
%/\/\/\/\/\/\/\/\/\/\/\/\/\/\/\/\/\/\/\/\/\/\/\/\/\/\/\/\/\/\/\/\/\/\/\/\
%							Schriftarten fertig
%------------------------------------------------------------------------
%________________________________________________________________________
%
\usepackage{rotating}%
\usepackage{setspace}%
\usepackage{framed}%
\usepackage{float}%
\usepackage{listings}%
% \usepackage{amsmath}%
\usepackage{mathtools}%
\usepackage{calc}%
\usepackage{ifthen}%
\usepackage{enumitem}%
\usepackage{xcolor}%
\usepackage{nameref}%
\usepackage{caption}%
\usepackage{graphicx}%
\usepackage{relsize}%
\usepackage{diagbox}%
\usepackage{multirow}%
\usepackage{colortbl}%
\usepackage{printlen}%
\usepackage{nicefrac}%
\usepackage{xparse}%
% \usepackage{xr}%
% \externaldocument{AbstractInterfaceProjektMain}%
% --------------------------------------------------------
%		Put this in the Header from Docs, where you want
%		to Import labels form other Docs (e.g. the main)
% --------------------------------------------------------
% \usepackage{zref-xr,zref-user}%
% \zxrsetup{toltxlabel=true,verbose=true}%
% \zexternaldocument*{AbstractInterfaceMainComplete}%
% --------------------------------------------------------
%		End for label Import
% --------------------------------------------------------
% See PassOptions for url on Top
\usepackage[%
	colorlinks=true,%
	urlcolor=blue,%
	linkcolor=black,%
% 	citecolor=citegreen,%
	citecolor=black,%
	filecolor=black,%
	breaklinks,%
	]{hyperref}%
\usepackage{url}%
\usepackage{nameref}%
\usepackage{cleveref}%
%
%
%
% \newcommand{\includestandalonedefaultmode}{buildnew}%
% \usepackage[%
% 	group=true,%
% 	mode=\includestandalonedefaultmode,%
% % 	subpreambles=true,%
% % 	build={%
% % 		latexoptions=-interaction=batchmode -shell-escape -jobname ’\buildjobname’%
% % 	},%
% 	]{standalone}%
%
%
%

% Makro to define a chapter without numbering, but with Entry in Table of Contents
% Chapter without numbering to TOC
% Arguments:
% #1 - Optional. Alternative Name for TOC
% #2 - Mandatory. Name of the Chap/Sec/Subsec itself
% #3 - Optional. Label
%
%					%				Old Version					%
% 					% 					% \newcommand{\chapnt}[2]{%
% 					% 					% 	\chapter*{#1}%
% 					% 					% 	\label{#2}%
% 					% 					% 	\addcontentsline{toc}{chapter}{\nameref{#2}}%
% 					% 					% 	\markboth{\nameref{#2}}{\nameref{#2}}%
% 					% 					% 	\addtocounter{chapter}{1}%
% 					% 					% }%
\DeclareDocumentCommand{\chapnt}{%
O{#2} m o%
}{%
	\chapter*{#2}%
	\IfValueTF{#3}{%
		\label{#3}%
	}{}%
% 	\addcontentsline{toc}{chapter}{\nameref{#3}}%
% 	\markboth{\nameref{#2}}{\nameref{#3}}%
	\addcontentsline{toc}{chapter}{#1}%
	\markboth{#1}{#1}%
	\addtocounter{chapter}{1}%
}%
% Usage: \chapnt{Chapter Name}{Label}
%
% And for Section
%	Sets the \rightmark. Remember to change the Brackets for \leftmark if needed
%	With empty Bracket it sets it blank
%					%				Old Version					%
% 					% 					% \newcommand{\secnt}[2]{%
% 					% 					% 	\section*{#1}%
% 					% 					% 	\label{#2}%
% 					% 					% 	\addcontentsline{toc}{section}{\nameref{#2}}%
% 					% 					% 	\markboth{}{\nameref{#2}}%
% 					% 					% }%
\DeclareDocumentCommand{\secnt}{%
O{#2} m o%
}{%
	\section*{#2}%
	\IfValueTF{#3}{%
		\label{#3}%
	}{}%
	\addcontentsline{toc}{section}{#1}%
% 	\markboth{\leftmark}{#1}%
	\markright{#1}%
}%
%
% And for SubSection
%					%				Old Version					%
% 					% 					% \newcommand{\subsecnt}[2]{%
% 					% 					% 	\subsection*{#1}%
% 					% 					% 	\label{#2}%
% 					% 					% 	%\addcontentsline{toc}{subsection}{\protect\numberline{}#1}%This would add a whitespace, where normally the section-numbering would be.
% 					% 					% 	\addcontentsline{toc}{subsection}{#1}%
% 					% 					% 	\markboth{}{\nameref{#2}}%
% 					% 					% }%
\DeclareDocumentCommand{\subsecnt}{%
O{#2} m o%
}{%
	\subsection*{#1}%
	\IfValueTF{#3}{%
		\label{#3}%
	}{}%
	%\addcontentsline{toc}{subsection}{\protect\numberline{}#1}%This would add a whitespace, where normally the section-numbering would be.
	\addcontentsline{toc}{subsection}{#1}%
}%
%
% Chapter with Subtitle
\newcommand\chaptersubt[2]{\chapter%
  [#1\hfil\hbox{}\protect\linebreak{\itshape#2}]%
  {#1\\[2ex]\Large\itshape#2}%
}
\newcommand\sectionsubt[2]{\section%
  [#1\hfil\hbox{}\protect\linebreak{\itshape#2}]%
  {#1\\[2ex]\Large\itshape#2}%
}
%
%
%
%
%
%
% own command to get a newline, without this disturbing underfull \hbox
% still beware of the "Glue" from the set \parskip height plus x minus y
% which in fact allows latex to stretch the space between paragraphs
% (by default up to 1pt)
\newcommand{\nl}{\par\noindent} % Abk. für NewLine
% and here a command to create really a new paragraph
% inserts an additional free line
\newcommand{\np}{\par\vspace{\baselineskip}} % Abk. für NewParagraph
\newcommand{\npi}{\par\vspace{\baselineskip}\noindent} % Abk. für NewParagraph_noIndent
% Zu benutzen also so:
% Neue Zeile ohne Einrückung: \nl
% Neue Zeile mit Einrückung: Freie Zeile im Editor
% Neuer Absatz mit freier Zeile dazwischen: \np
\newcommand{\greenuline}[1]{\colorlet{temp}{.}\color{green}\underline{\color{temp}#1}\color{temp}\xspace}%
%
%
%
%
%WorkingStateDivider
\newcommand{\wdiv}{\vspace{0.5\baselineskip}\npi\leavevmode\xleaders\hbox{=}\hfill\kern0pt\nl\#\#\#\#\#\#\#\#\#\#\#\#\#\#\#\#\#\#\#\#\#\#\#\#\leavevmode\xleaders\hbox{ ・ }\hfill\kern0pt\#\#\#\#\#\#\#\#\#\#\#\#\#\#\#\#\#\#\#\#\#\#\#\#\nl}
%
%
%
% Eventuell Labelprüfung vor \cref@gettype einbauen:
%	(ifcsdef aus der etoolbox)
% Makro zur einheitlichen Formatierung von Abbildungs-Referenzen
\newcommand{\abbref}[1]{Abbildung~\ref{#1}}%
\newcommand{\abbrefzwei}[2]{Abbildungen~\ref{#1} \& \ref{#2}}%
% Und das gleiche inklusive \nameref
% Makro zur einheitlichen Formatierung von Abschnitts-Referenzen
\newcommand{\abnamref}[1]{Abbildung~\ref{#1} (\nameref{#1})}%
\newcommand{\absref}[1]{\cref{#1}}%
% \newcommand{\absref}[1]{\textbf{Abschnitt~\ref{#1}}}%
% Und das gleiche inklusive \nameref
\makeatletter%
\newcommand{\absamref}[1]{%
 	\cref@gettype{#1}{\currlabtype}%
	\IfSubStr{\currlabtype}{part}{%
		\textsf{\cref{#1} - \nameref{#1}}%
	}{%
	\IfSubStr{#1}{subsubsec}{%
%	'subsubsection' isn't functional with gettype.
%	This only goes deep until subsection
%	So be careful and append a preceding 'subsubsec'
%	on the label
		Unterabschnitt >>\nameref{#1}<< aus \ref{#1}%
	}{%
		\cref{#1} (\nameref{#1})%
	}%
	}%
}%
\makeatother%
% \newcommand{\absamref}[1]{\textbf{Abschnitt~\ref{#1}} (\nameref{#1})}%
% Für 2 Abschnitte zugleich
\makeatletter%
\newcommand{\absrefzwei}[2]{%
	\cref@gettype{#1}{\currlabtype}%
	\IfSubStr{\currlabtype}{chapter}{%
		Kapiteln~\ref{#1} \& \ref{#2}%
	}{%
		Abschnitten~\ref{#1} \& \ref{#2}%
	}%
}%
\makeatother%
\makeatletter%
\newcommand{\absamrefzwei}[2]{%
	\cref@gettype{#1}{\currlabtype}%
	\IfSubStr{\currlabtype}{chapter}{%
		Kapiteln~\ref{#1} (\nameref{#1}) \&
			\ref{#2} (\nameref{#2})%
	}{%
		Abschnitten~\ref{#1} (\nameref{#1}) \&
			\ref{#2} (\nameref{#2})%
	}%
}%
\makeatother%
% Absatz Name Reference Zwei Short -- means without the subsequent 'n' -
% i.e. Kapitel instead of Kapiteln
\makeatletter%
\newcommand{\absamrefzweis}[2]{%
	\cref@gettype{#1}{\currlabtype}%
	\IfSubStr{\currlabtype}{chapter}{%
		Kapitel~\ref{#1} (\nameref{#1}) \&
			\ref{#2} (\nameref{#2})%
	}{%
		Abschnitte~\ref{#1} (\nameref{#1}) \&
			\ref{#2} (\nameref{#2})%
	}%
}%
\makeatother%
% \newcommand{\absrefzwei}[2]{\textbf{Abschnitten~\ref{#1}} \& \ref{\ref{#2}}}%
% \newcommand{\absamrefzwei}[2]{\textbf{Abschnitten~\ref{#1}} (\nameref{#1}) \& \textbf{\ref{#2}} (\nameref{#2})}%
%
% 
%
%
%
% 
% Makro für Kreis um Text
\newcommand\circlearound[1]{%
  \tikz[baseline]\node[draw,shape=circle,anchor=base,inner sep=0.3ex] {#1} ;}
\newcommand\lipsaround[1]{%
  \tikz[baseline]\node[draw,shape=ellipse,anchor=base,inner sep=0.3ex] {#1} ;}
%
\newcommand\rectdotaround[2]{%
  \tikz[baseline]\node[draw=#1,thick, loosely dotted,%
  shape=rectangle,anchor=base,inner sep=0.5ex] {#2} ;}
%
%
%
%
%Some Symbols. (Using package 'amssymb', 'pifont')
\newcommand{\checksign}{\ding{51}}%
\newcommand{\crosssign}{\ding{55}}%
%
%
%
%
%
%
%=================================
% The romaji Spoiler-Environment
%      requires package 'ocgx' / 'ocgx2'
%================================================
%	Button at Start of Env
\newcounter{romaji}%
\newenvironment{romaji}%
{\stepcounter{romaji}%
\begin{minipage}{\textwidth-1ex}%
\switchocg{romaji\arabic{romaji}}{\fbox{\tiny ローマ字}}\begin{ocg}{Romaji \arabic{romaji}}{romaji\arabic{romaji}}{0}%
\footnotesize}%
%---------------------------------------------
{\par\end{ocg}%
\end{minipage}}%
%================================================
%	Button at End of Env
% 		\newcounter{romaji}%
% 		\newenvironment{romaji}%
% 		{\stepcounter{romaji}%
% 		\begin{minipage}{\textwidth-1ex}%
% 		\begin{ocg}{Romaji \arabic{romaji}}{romaji\arabic{romaji}}{0}%
% 		\footnotesize}%
% 		%---------------------------------------------
% 		{\end{ocg}\switchocg{romaji\arabic{romaji}}{\fbox{\tiny ローマ字}}%
% 		\end{minipage}}%
%================================================
\newcommand{\ropre}{\par\vspace{0.2\baselineskip}\noindent}% Romaji-Pre
%
%
%
%
%
%
%
%
%
%
%
%
%
%
%________________________________________________________________________
%------------------------------------------------------------------------
%			Japanische Schrift
%/\/\/\/\/\/\/\/\/\/\/\/\/\/\/\/\/\/\/\/\/\/\/\/\/\/\/\/\/\/\/\/\/\/\/\/\
% \makeatletter%
% \newcommand{\jap}[1]{\begin{CJK}{UTF8}{min}#1\end{CJK}}%
% \makeatother%
%/\/\/\/\/\/\/\/\/\/\/\/\/\/\/\/\/\/\/\/\/\/\/\/\/\/\/\/\/\/\/\/\/\/\/\/\
%							Japanisch fertig
%------------------------------------------------------------------------
%________________________________________________________________________
%
%
%
%________________________________________________________________________
%------------------------------------------------------------------------
%			Verschiedene Definitionen wie Schriftzeichen
%/\/\/\/\/\/\/\/\/\/\/\/\/\/\/\/\/\/\/\/\/\/\/\/\/\/\/\/\/\/\/\/\/\/\/\/\
\newcommand{\mytilde}{{\raise.17ex\hbox{$\scriptstyle\mathtt{\sim}$}}}
\newcommand{\mytildeA}{\textasciitilde}
\newcommand{\mytildeB}{$\sim$}
\newcommand{\mytildeC}{\~{}}
\newcommand{\mytildeD}{\texttildelow}
\newcommand{\mytildeE}{\raisebox{-.8ex}{\textasciitilde}}
%/\/\/\/\/\/\/\/\/\/\/\/\/\/\/\/\/\/\/\/\/\/\/\/\/\/\/\/\/\/\/\/\/\/\/\/\
%							Definitionen fertig
%------------------------------------------------------------------------
%________________________________________________________________________
%
%
%
%
%
%
%
%
%
% Optionen für fontseries und fontshape:
% Series, any combination of weight and width is [in theory] possible:
% weight                    width
% Ultra Light       ul      Ultra Condensed     uc    
% Extra Light       el      Extra Condensed     ec    
% Light             l       Condensed            c      
% Semi Light        sl      Semi Condensed      sc    
% Medium (normal)   m
% Semi Bold         sb      Semi Expanded       sx    
% Bold              b       Expanded             x 
% Extra Bold        eb      Extra Expanded      ex 
% Ultra Bold        ub      Ultra Expanded      ux
% %
% Shape:
% upright (normal)   n 
% italic             it
% slanted/oblique    sl 
% small caps         sc
% upright italic     ui
% outline            ol 
%
%Inline English
%Zum anders formatieren von nicht übersetzten, englischen Begriffen
\newcommand*{\ien}[1]{%
	#1%
% 	{\small\fontfamily{phv}\fontseries{m}\fontshape{n}\selectfont#1}%
}%
%
%Inline Code
%Zum eigens formatieren von kleinen Code-Fetzen
\newcommand*{\ico}[1]{%
	\texttt{#1}%
}%
%
%
%
%
% Kommando-Zeilen Syntax
% The one with the 'b' suffix is for brackets only. It does not format anything,
% but add the brackets. Its something like for "inner-use". You can use it in
% wrapped makros for example.
\newcommand*{\regexcmd}[2]{% Regular Expression Command Line
	\IfSubStr{#1}{id}{%
		\texttthyph{\textcolor{ColorRegExCmdID}{<#2>}}%
	}{\IfSubStr{#1}{auswahl}{%
		\texttthyph{\textcolor{ColorRegExCmdSelect}{[#2]}}%
	}{\IfSubStr{#1}{mehrfach}{%
		\texttthyph{\textcolor{ColorRegExCmdMultiSelect}{\{#2\}}}%
	}{\IfSubStr{#1}{option}{%
		\texttthyph{-\textcolor{ColorRegExCmdOption}{$\|$#2$\|$}}%
	}{%
	}}}}%
}%
\newcommand*{\regexcmdb}[2]{% Regular Expression Command Line
	\IfSubStr{#1}{id}{%
		<#2>%
	}{\IfSubStr{#1}{auswahl}{%
		[#2]%
	}{\IfSubStr{#1}{mehrfach}{%
		\{#2\}%
	}{\IfSubStr{#1}{option}{%
		-$\|$#2$\|$%
	}{%
	}}}}%
}%
%
%
%
%
%
\definecolor{marginparcolor}{RGB}{31,73,125}%
% Neuer MarginPar, für Schriftgröße
\newcommand*{\Marginpar}{}% Anweisung "reservieren"
% Das "Standard-Kommando" in neuem Kommando mit großem 'M' retten:
\let\Marginpar\marginpar% \MarginPar ist jetzt dasselbe wie \marginpar
\renewcommand*{\marginpar}[2][]{{% ein optionales und ein normales Argument
\expandafter\renewcommand*{\glstextformat}[1]{\color{marginparcolor}##1}%
  \ifstr{#1}{}{%
  	\Marginpar{\footnotesize\raggedright \textcolor{marginparcolor}{#2}}%
  }{%
    \Marginpar[{\footnotesize\raggedright \textcolor{marginparcolor}{#1}}]%
    	{\footnotesize\raggedright \textcolor{marginparcolor}{#2}}%
  }%
}}% Double Braces to create a scope for the redefinition of glstextformat
%
%
%
%
%
%________________________________________________________________________
%------------------------------------------------------------------------
%							Makros für Cleveref
%/\/\/\/\/\/\/\/\/\/\/\/\/\/\/\/\/\/\/\/\/\/\/\/\/\/\/\/\/\/\/\/\/\/\/\/\
%Get Type of \ref with help of cleveref
\def\currlabtype{}% Variable reservieren, um im ChapRef Makro den Typ zu speichern
%/\/\/\/\/\/\/\/\/\/\/\/\/\/\/\/\/\/\/\/\/\/\/\/\/\/\/\/\/\/\/\/\/\/\/\/\
%							Cleveref fertig
%------------------------------------------------------------------------
%________________________________________________________________________
%
%
%
%
%
\newcommand{\tikzpicturescale}{1.0}
\newcommand{\tikzpicturescaleboxfactor}{1.0}
\newcommand{\stdleftbarwidth}{0.2em}
% 
%
%
%
%
% INFO:
% To print a value (length, printlength)
% \the\value
% e.g.: \the\textwidth
% e.g.: \the\linewidth
% e.g.: \the\columnwidth
%
%
%
%
% tikz makros
%
% Variable für die Breite/Höhe von eingebundenen Bildern
% Verwendet im folgenden Makro
\newlength{\tikzwidth}%
\newlength{\tikzheight}%
\newlength{\textheightscaled}%
\newlength{\textwidthactual}%
\newcommand{\tikzHeightFactorArgument}{1}%
\newcommand{\tikzWidthFactorArgument}{1}%
\newlength{\tikzAimedHeight}%
\newlength{\tikzAimedWidth}%
\newlength{\captionHeight}%
\newcommand{\tikzFloatPositioning}{!ht}
% \textheightscaled=0.96\textheight%
% Makro für die Tikz Abbildungen
% Arg#1 (Optional): Includestandalone-Mode
% Arg#2: Pic-Name (Location: \tikzFilesPath/NAME.tex) % This Macro can be defined somewhere, by you. In my Template it is found inside the Preamble right by the inclusion of the tikz-Package, together with \includestandalonedefaultmode
% Arg#3 (Optional): Alternatives Caption (TableOfContents usw.)
% Arg#4: Bildbeschriftung
% Arg#5: Number of Lines des Caption
% Arg#6 (Optional): Liste mit folgendem Inhalt. Parameter zu Skalierung und Positionierung
%			#1: Faktor der Seitenhöhe, der nicht überschritten werden soll 
%			#2: Faktor der Seitenbreite, der nicht überschritten werden soll
%			#3: Positionierungsmodus. Bsp.: !ht, !htpb, H
%	  Beispiel für ein Listen-Format: [1,1,!htp]
% Arg#7: Label für referencing
% Arg#8 (Optional): Rotations-Winkel für das Bild
%% Möglichkeiten für den includestandalone-mode: tex, buildnew
%% The Default Macro: See Definition in Header (should be buildnew)
%% Use the optinal Argument with "tex", to make the tikz-Pics inline and hence
%% compiled every Runtime, without pre-rendered picture. buildnew creates
%% pre-rendered Pictures and includes them like standard-png; only rendered, if
%% tex-file is newer than the picture
\DeclareDocumentCommand{\tikzabb}{%
O{\includestandalonedefaultmode} m o m m >{\SplitList{,}}O{1,1,!ht} m O{0}%
}{%
\def\efigure{\begin{figure}}%
\def\efigureend{\end{figure}}%
\SplitListScalePos#6\relax%
%\textheightscaled=\textheight-\parskip-\abovecaptionskip-\belowcaptionskip-\baselineskip%
\textheightscaled=\dimexpr\textheight-#5\baselineskip-\parskip-\abovecaptionskip-\belowcaptionskip\relax%
\textwidthactual=\dimexpr\linewidth\relax%
\ifthenelse{#8=90}{%
	\tikzAimedHeight=\tikzWidthFactorArgument\textwidthactual%
	\tikzAimedWidth=\tikzHeightFactorArgument\textheightscaled%
}{
	\ifthenelse{#8=-90}{%
		\tikzAimedHeight=\tikzWidthFactorArgument\textwidthactual%
		\tikzAimedWidth=\tikzHeightFactorArgument\textheightscaled%
	}{
		\tikzAimedHeight=\tikzHeightFactorArgument\textheightscaled%
		\tikzAimedWidth=\tikzWidthFactorArgument\textwidthactual%
	}%
}%
\settowidth{\tikzwidth}{\includestandalone{\tikzFilesPath#2}}%
\settoheight{\tikzheight}{\includestandalone{\tikzFilesPath#2}}%
\tikzwidth=\tikzpicturescaleboxfactor\tikzwidth%
\tikzheight=\tikzpicturescaleboxfactor\tikzheight%
\ifthenelse{\tikzAimedHeight<\tikzheight}{%
	\tikzheight=\tikzAimedHeight%
	\settowidth{\tikzwidth}{\includestandalone[height=\tikzheight]{\tikzFilesPath#2}}%
	\ifthenelse{\tikzAimedWidth<\tikzwidth}{%
		\tikzwidth=\tikzAimedWidth%
		\expandafter\efigure\expandafter[\tikzFloatPositioning]%
			\centering%
% 	 		\fbox{%
			\includestandalone[width=\tikzwidth,angle=#8,mode=#1]{\tikzFilesPath#2}%
% 	 		}%
			\IfValueTF{#3}{%
				\caption[#3]{#4}%
			}{%
				\caption{#4}%
			}%
			\label{#7}%
		\efigureend%
	}{%
		\expandafter\efigure\expandafter[\tikzFloatPositioning]%
			\centering%
% 	 		\fbox{%
			\includestandalone[height=\tikzheight,angle=#8,mode=#1]{\tikzFilesPath#2}%
% 	 		}%
			\IfValueTF{#3}{%
				\caption[#3]{#4}%
			}{%
				\caption{#4}%
			}%
			\label{#7}%
		\efigureend%
	}%
}{%
	\ifthenelse{\tikzAimedWidth<\tikzwidth}{\tikzwidth=\tikzAimedWidth}{}%
	\expandafter\efigure\expandafter[\tikzFloatPositioning]%
		\centering%
% 		\fbox{%
		\includestandalone[width=\tikzwidth,mode=#1]{\tikzFilesPath#2}%
% 		}%
		\IfValueTF{#3}{%
			\caption[#3]{#4}%
		}{%
			\caption{#4}%
		}%
		\label{#7}%
	\efigureend%
}%
}%
%
\newcommand{\SplitListScalePos}[3]{%
	\renewcommand{\tikzHeightFactorArgument}{#1}%
	\renewcommand{\tikzWidthFactorArgument}{#2}%
	\renewcommand{\tikzFloatPositioning}{#3}%
}%
%
% Nice Makro, which allows to check if a node with certain name exists.
% Gives you an If-Then-Else Operator for use like this:
% \ifnodedefined{Node-name without brackets}{Wenn existent, dann das}{else}
\makeatletter%
\long\def\ifnodedefined#1#2#3{%
  \@ifundefined{pgf@sh@ns@#1}{#3}{#2}}%
%
\newcommand\aeundefinenode[1]{%%
  \expandafter\ifx\csname pgf@sh@ns@#1\endcsname\relax%
  \else%
    \typeout{===>Undefining node "#1"}%
    \global\expandafter\let\csname pgf@sh@ns@#1\endcsname\relax%
  \fi%
}%
%Another to undefine nodes, that they are gone in successive pics
\newcommand\aeundefinethesenodes[1]{%
  \foreach \myn  in {#1}%
    {%
      \expandafter\aeundefinenode\expandafter{\myn}%
    }%
}%
\makeatother%
%
%
% Own redefinition of leftbar environment
% Einfach ums schöner zu machen
\newenvironment{myleftbar}[2]%[1=0.5pt,2=5pt]%
{\setlength{\topsep}{0ex}%
\def\FrameCommand{\hspace{#2} \vrule width #1 \hspace{0.0em}}%
\MakeFramed%
{\advance\hsize-\width \FrameRestore}}%
{\endMakeFramed}%
%
%
\newenvironment{scope}{}{}
%
%
%
%
% Own Makro für footnote ohne Marker
\newcommand\footnoteblank[1]{%
  \begingroup
  \renewcommand\thefootnote{}\footnote{#1}%
  \addtocounter{footnote}{-1}%
  \endgroup
}
%
%
%
%
%
%
%==========================================
%##########################################
%    Schriftgrößen
%==========================================
\tiny
\scriptsize
\footnotesize
\small
\normalsize
\large
\Large
\LARGE
\huge
\Huge
%##########################################
%==========================================%
%
% Extrusion Makros (needs shape=rectangleRoundedAnchorSurround)
% You have to specify a coordinate-Node as Base-Anchor and pass it
% Then pass the Directions
% Styles have to contain at least:
%	minimum width, minimum height, rounded corners attributes=RoundedCornersAmount
% Arguments:
% {AnchorNode}
% {up / down} %Not used for now
% {left / right} %Not used for now
% {StyleFront}
% {StyleBack}
% {FrontColor}
% {BackColor}
% {RoundedCornersAmount}
% Later on added:
% {ShiftAmountX}
% {SjoftAmountY}
%
\DeclareDocumentCommand{\ExtrudeOutRoundedTop}{ m m m m m m m m }{%
	\node(#1back)[#5,fill=#7,anchor=south]at(#1){};%
	\node(#1front)[#4,fill=#6,xshift=-2mm,yshift=2mm,anchor=south]at(#1){};%
	\path[draw=black,fill=#7](#1front.south west)--(#1front.south east)--%
		(#1back.south east)-- (#1back.south west)--(#1front.south west);%
	\path[shading=axis,%
		top color=#6,bottom color=#7,%
% 		middle color=#6,%
		shading angle=45,%
		transform canvas={shift={(-0.5\pgflinewidth,0)}},%
		]%
		($(#1front.east|-#1front.north)+(-#8,0)$)to[out=0,in=90]%
		($(#1front.east|-#1front.north)+(0,-#8)$)to%
		($(#1back.east|-#1back.north)+(0,-#8)$)to[out=90,in=0]%
		($(#1back.east|-#1back.north)+(-#8,0)$)to%
		($(#1front.east|-#1front.north)+(-#8,0)$);%
% 	\path[draw=black]%
% 		($(#1front.east|-#1front.north)+(-#8,0)$)to[out=0,in=90]%
% 		($(#1front.east|-#1front.north)+(0,-#8)$);%
% 	\path[draw=black]%
% 		($(#1back.east|-#1back.north)+(0,-16pt)$)to%
% 		($(#1back.east|-#1back.north)+(0,-#8)$)to[out=90,in=-45]%
%  		(#1back.north east);%
% 		($(#1back.east|-#1back.north)+(-#8,0)$)to%
% 		($(#1back.east|-#1back.north)+(-16pt,0)$);%
	\node(#1front2)[#4,fill=#6,anchor=south]at(#1front.south){};%
}%
%
\DeclareDocumentCommand{\ExtrudeOutRoundedBottom}{ m m m m m m m m }{%
	\node(#1back)[#5,fill=#7,anchor=north]at(#1){};%
	\node(#1front)[#4,fill=#6,xshift=-2mm,yshift=2mm,anchor=north]at(#1){};%
	\path[fill=abifacebackcolor](#1front.north east)--%
		(#1back.north east)--%
		(#1back.north east-|#1front.north east)--%
		(#1front.north east);%
		\draw[draw=abifacebackcolor,line width=0.8pt]%
			($(#1back.north east)+(0,-0.25\pgflinewidth)$)--%
			($(#1back.north east-|#1front.north east)+(0,-0.25\pgflinewidth)$);%
		\draw(#1front.north east)--(#1back.north east);%
		\draw[](#1back.north east)--%
			($(#1back.north east)+(0,-0.6pt)$);%
	\path[shading=axis,%
		left color=abifacebackcolor,right color=abifacebackcolor,%
		middle color=abifacecolor,%
		shading angle=135,]%
		($(#1front.east|-#1front.south)+(0pt,#8)$)to[out=-90,in=0]%
		($(#1front.east|-#1front.south)+(-#8,0pt)$)to%
		($(#1back.east|-#1back.south)+(-#8,0pt)$)to[out=0,in=-90]%
		($(#1back.east|-#1back.south)+(0pt,#8)$)to%
		($(#1front.east|-#1front.south)+(0pt,#8)$);%
	\path[draw=black]%
		($(#1back.east|-#1back.south)+(-16pt,0pt)$)to%
		($(#1back.east|-#1back.south)+(-#8,0pt)$)to[out=0,in=-90]%
		($(#1back.east|-#1back.south)+(0pt,#8)$)to%
		($(#1back.east|-#1back.south)+(0pt,16pt)$);%
	\path[shading=axis,%
		bottom color=#6,top color=#7,%
% 		middle color=#6,%
		shading angle=-135,%
		transform canvas={shift={(0,0.5\pgflinewidth)}},%
		]%
		($(#1front.west|-#1front.south)+(0,#8)$)to[out=-90,in=0]%
		($(#1front.west|-#1front.south)+(#8,0)$)to%
		($(#1back.west|-#1back.south)+(#8,0)$)to[out=180,in=-90]%
		($(#1back.west|-#1back.south)+(0,#8)$)to%
		($(#1front.west|-#1front.south)+(0,#8)$);%
	\node(#1front2)[#4,fill=#6,anchor=north]at(#1front.north){};%
}%%
%
%
%
%
%
\usepackage{tikz}%
\usetikzlibrary{calc}%
\usetikzlibrary{positioning}%
\usetikzlibrary{shapes}%
\usetikzlibrary{shapes.geometric}%
\usetikzlibrary{decorations}%
\usetikzlibrary{decorations.markings}%
\usetikzlibrary{decorations.pathmorphing}%
\usetikzlibrary{shadows}%
\usetikzlibrary{shadows.blur}%
\usetikzlibrary{arrows}%
\usetikzlibrary{arrows.meta}%
\usetikzlibrary{fadings}%
\usetikzlibrary{shadings}%
\usetikzlibrary{matrix}%
\usetikzlibrary{fit}%
\usepackage{pgfplots}%
\pgfplotsset{compat=1.11}%
%
%
%
%
%
% Info:
% Linienbreiten (line width)(\pgflinewidth):
% \tikzset{
%     ultra thin/.style= {line width=0.1pt},
%     very thin/.style=  {line width=0.2pt},
%     thin/.style=       {line width=0.4pt},% thin is the default
%     semithick/.style=  {line width=0.6pt},
%     thick/.style=      {line width=0.8pt},
%     very thick/.style= {line width=1.2pt},
%     ultra thick/.style={line width=1.6pt}
% }
%
%
% Line-Patern
% 	\tikzstyle{solid}=                   [dash pattern=]
% 	\tikzstyle{dotted}=                  [dash pattern=on \pgflinewidth off 2pt]
% 	\tikzstyle{densely dotted}=          [dash pattern=on \pgflinewidth off 1pt]
% 	\tikzstyle{loosely dotted}=          [dash pattern=on \pgflinewidth off 4pt]
% 	\tikzstyle{dashed}=                  [dash pattern=on 3pt off 3pt]
% 	\tikzstyle{densely dashed}=          [dash pattern=on 3pt off 2pt]
% 	\tikzstyle{loosely dashed}=          [dash pattern=on 3pt off 6pt]
% 	\tikzstyle{dashdotted}=              [dash pattern=on 3pt off 2pt on \the\pgflinewidth off 2pt]
% 	\tikzstyle{densely dashdotted}=      [dash pattern=on 3pt off 1pt on \the\pgflinewidth off 1pt]
% 	\tikzstyle{loosely dashdotted}=      [dash pattern=on 3pt off 4pt on \the\pgflinewidth off 4pt]
%